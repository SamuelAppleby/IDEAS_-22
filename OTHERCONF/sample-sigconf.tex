%%
%% This is file `sample-sigconf.tex',
%% generated with the docstrip utility.
%%
%% The original source files were:
%%
%% samples.dtx  (with options: `sigconf')
%% 
%% IMPORTANT NOTICE:
%% 
%% For the copyright see the source file.
%% 
%% Any modified versions of this file must be renamed
%% with new filenames distinct from sample-sigconf.tex.
%% 
%% For distribution of the original source see the terms
%% for copying and modification in the file samples.dtx.
%% 
%% This generated file may be distributed as long as the
%% original source files, as listed above, are part of the
%% same distribution. (The sources need not necessarily be
%% in the same archive or directory.)
%%
%%
%% Commands for TeXCount
%TC:macro \cite [option:text,text]
%TC:macro \citep [option:text,text]
%TC:macro \citet [option:text,text]
%TC:envir table 0 1
%TC:envir table* 0 1
%TC:envir tabular [ignore] word
%TC:envir displaymath 0 word
%TC:envir math 0 word
%TC:envir comment 0 0
%%
%%
%% The first command in your LaTeX source must be the \documentclass command.
\documentclass[sigconf]{acmart}
\let\comment\undefined		% Undefine amsmath comment, macro confict otherwise
\usepackage{easyReview}
\usepackage{savesym}
\usepackage{xcolor}
\usepackage{booktabs}
\usepackage{multirow}
%\usepackage[english]{babel}
%\usepackage{environ}
%\usepackage{newclude}
%\usepackage{rotating}
%\usepackage{pdflscape}
\usepackage{adjustbox}
%%% Pseudocodes
%\usepackage{inconsolata}
%\usepackage{floatrow}
%\newfloatcommand{capbtabbox}{table}[][\FBwidth]
\usepackage{graphicx}
\usepackage{caption}
%\usepackage{subcaption}

\usepackage{tabularx}% added for table design
\usepackage[noend]{algpseudocode}
%% Multiline
\newcommand\CONDITION[2]%
{\begin{tabular}[t]{@{}l@{}l@{}}
		#1&#2
	\end{tabular}%
}
\algdef{SE}[WHILE]{While}{EndWhile}[1]%
{\algorithmicwhile\ \CONDITION{#1}{\ \algorithmicdo}}%
{\algorithmicend\ \algorithmicwhile}
\algdef{SE}[FOR]{For}{EndFor}[1]%
{\algorithmicfor\ \CONDITION{#1}{\ \algorithmicdo}}%
{\algorithmicend\ \algorithmicfor}
\algdef{S}[FOR]{ForAll}[1]%
{\algorithmicforall\ \CONDITION{#1}{\ \algorithmicdo}}
\algdef{SE}[REPEAT]{Repeat}{Until}{\algorithmicrepeat}[1]%
{\algorithmicuntil\ \CONDITION{#1}{}}
\algdef{SE}[IF]{If}{EndIf}[1]%
{\algorithmicif\ \CONDITION{#1}{\ \algorithmicthen}}%
{\algorithmicend\ \algorithmicif}%
\algdef{C}[IF]{IF}{ElsIf}[1]%
{\algorithmicelse\ \algorithmicif\ \CONDITION{#1}{\ \algorithmicthen}}
%% End Multiline
\usepackage{algorithm,algorithmicx}
%\makeatletter
%\algrenewcommand\ALG@beginalgorithmic{\ttfamily}
%\makeatother
\usepackage{adjustbox} %% Fitting to pageRichard
\usepackage{amsmath,stmaryrd}
%% mathematics
\usepackage{braket} %% Set, tuple...
\usepackage{pifont,bigdelim}
\usepackage{amsmath,amssymb}
\newcommand{\Next}{\ensuremath{\bigcirc}}
\newcommand{\Globally}{\ensuremath{\Box}}
\newcommand{\Future}{\ensuremath{\lozenge}}
\newcommand{\PopUpComment}[2]{\pdfcomment[author={#1}]{#2}}
\newcommand{\DeclareClause}[5]{\textsf{#1}(\texttt{#2},#3,\texttt{#4},#5)}
\newcommand{\DeclareClauseWithJoin}[6]{\textsf{#1}(\texttt{#2},#3,\texttt{#4},#5)\;\textsf{where}\;#6}
\newcommand{\LTLf}{\textup{LTL}\textsubscript{f}\;}

%%
%% \BibTeX command to typeset BibTeX logo in the docs
\AtBeginDocument{%
  \providecommand\BibTeX{{%
    Bib\TeX}}}

%% Rights management information.  This information is sent to you
%% when you complete the rights form.  These commands have SAMPLE
%% values in them; it is your responsibility as an author to replace
%% the commands and values with those provided to you when you
%% complete the rights form.
\setcopyright{acmcopyright}
\copyrightyear{2022}
\acmYear{2022}
\acmDOI{XXXXXXX.XXXXXXX}
%% These commands are for a PROCEEDINGS abstract or paper.
\acmConference[IDEAS'22]{26th International Database Application \& Enginnerring Symosium}{August 22--24,
  2022}{Budapest, HU}
\acmPrice{15.00}
\acmISBN{978-1-4503-XXXX-X/18/06}


%%
%% Submission ID.
%% Use this when submitting an article to a sponsored event. You'll
%% receive a unique submission ID from the organizers
%% of the event, and this ID should be used as the parameter to this command.
%%\acmSubmissionID{123-A56-BU3}

%%
%% For managing citations, it is recommended to use bibliography
%% files in BibTeX format.
%%
%% You can then either use BibTeX with the ACM-Reference-Format style,
%% or BibLaTeX with the acmnumeric or acmauthoryear sytles, that include
%% support for advanced citation of software artefact from the
%% biblatex-software package, also separately available on CTAN.
%%
%% Look at the sample-*-biblatex.tex files for templates showcasing
%% the biblatex styles.
%%

%%
%% The majority of ACM publications use numbered citations and
%% references.  The command \citestyle{authoryear} switches to the
%% "author year" style.
%%
%% If you are preparing content for an event
%% sponsored by ACM SIGGRAPH, you must use the "author year" style of
%% citations and references.
%% Uncommenting
%% the next command will enable that style.
%%\citestyle{acmauthoryear}



%%
%% end of the preamble, start of the body of the document source.
\begin{document}

%%
%% The "title" command has an optional parameter,
%% allowing the author to define a "short title" to be used in page headers.
	\title{Running Temporal Logical Queries on Knowledge Bases}

%%
%% The "author" command and its associated commands are used to define
%% the authors and their affiliations.
%% Of note is the shared affiliation of the first two authors, and the
%% "authornote" and "authornotemark" commands
%% used to denote shared contribution to the research.
\author{Samuel Appleby}
\orcid{0000-0000-0000-0000}
\email{s.appleby3@newcastle.ac.uk}
\affiliation{%
	\institution{Newcastle University\\ School of Computing}
	\country{United Kingdom}
}


\author{Giacomo Bergami}
\orcid{0000-0002-1844-0851}
\email{ngb113@newcastle.ac.uk}
\affiliation{%
	\institution{Newcastle University\\ School of Computing}
	\country{United Kingdom}
}


\author{Graham Morgan}
\orcid{0000-0002-0089-0395}
\email{graham.morgan@newcastle.ac.uk}
\affiliation{%
	\institution{Newcastle University\\ School of Computing}
	\country{United Kingdom}
}

%%
%% By default, the full list of authors will be used in the page
%% headers. Often, this list is too long, and will overlap
%% other information printed in the page headers. This command allows
%% the author to define a more concise list
%% of authors' names for this purpose.
	\renewcommand{\shortauthors}{Appleby et al.}

%%
%% The abstract is a short summary of the work to be presented in the
%% article.
\begin{abstract}
State of the art for model checking exploit computationally intensive solutions, bottlenecked by either repeated data access or suboptimal algorithmic implementations. Our solution outperforms the previous solutions while proposing novel temporal logic operators for accessing relational tables.	
\end{abstract}

%%
%% The code below is generated by the tool at http://dl.acm.org/ccs.cfm.
%% Please copy and paste the code instead of the example below.
%%
\begin{CCSXML}
	<ccs2012>
	<concept>
	<concept_id>10002951.10003227.10003351.10003443</concept_id>
	<concept_desc>Information systems~Association rules</concept_desc>
	<concept_significance>500</concept_significance>
	</concept>
	<concept>
	<concept_id>10002951.10003227.10003351</concept_id>
	<concept_desc>Information systems~Data mining</concept_desc>
	<concept_significance>500</concept_significance>
	</concept>
	<concept>
	<concept_id>10002951.10002952.10002971.10003450.10010828</concept_id>
	<concept_desc>Information systems~Data scans</concept_desc>
	<concept_significance>300</concept_significance>
	</concept>
	<concept>
	<concept_id>10002951.10002952.10002971.10003450</concept_id>
	<concept_desc>Information systems~Data access methods</concept_desc>
	<concept_significance>300</concept_significance>
	</concept>
	<concept>
	<concept_id>10002951.10002952.10003190.10003192.10003210</concept_id>
	<concept_desc>Information systems~Query optimization</concept_desc>
	<concept_significance>500</concept_significance>
	</concept>
	<concept>
	<concept_id>10002951.10002952.10003190.10003192</concept_id>
	<concept_desc>Information systems~Database query processing</concept_desc>
	<concept_significance>500</concept_significance>
	</concept>
	<concept>
	<concept_id>10010147.10010178.10010187.10010193</concept_id>
	<concept_desc>Computing methodologies~Temporal reasoning</concept_desc>
	<concept_significance>300</concept_significance>
	</concept>
	</ccs2012>
\end{CCSXML}

\ccsdesc[500]{Information systems~Association rules}
\ccsdesc[500]{Information systems~Data mining}
\ccsdesc[300]{Information systems~Data scans}
\ccsdesc[300]{Information systems~Data access methods}
\ccsdesc[500]{Information systems~Query optimization}
\ccsdesc[500]{Information systems~Database query processing}
\ccsdesc[300]{Computing methodologies~Temporal reasoning}

%%
%% Keywords. The author(s) should pick words that accurately describe
%% the work being presented. Separate the keywords with commas.
\keywords{Logical Artificial Intelligence, Knowledge Bases, Query Plan, Temporal Logic}
%% A "teaser" image appears between the author and affiliation
%% information and the body of the document, and typically spans the
%% page.
%\begin{teaserfigure}
%  \includegraphics[width=\textwidth]{sampleteaser}
%  \caption{Seattle Mariners at Spring Training, 2010.}
%  \Description{Enjoying the baseball game from the third-base
%  seats. Ichiro Suzuki preparing to bat.}
%  \label{fig:teaser}
%\end{teaserfigure}

%%
%% This command processes the author and affiliation and title
%% information and builds the first part of the formatted document.
\maketitle

\section{Introduction}

\paragraph*{Brief introduction of your general area of interest: provide the \textbf{context} to the overall general setting.} \textit{Conformance checking} is an integral part of artificial intelligence bridging data mining and business process management \cite{bpm21}. It assesses whether a sequence of distinguishable events (i.e., a \textit{trace}) conforms to the expected process behaviour represented as a \textit{process model} \cite{RozinatA08}.These constraints can be expressed by either multiple declarative human readable templates in conjunction \cite{Li2020}, where each one of these represent a different temporal behavioural pattern, or equivalently as Finite State Machines \cite{AgostinelliBFMM21}. %Such a model might be either represented as a set of temporal clauses, determining correlations between events happening at a previous time of the trace (\textit{activation}) and others happening in the immediate future (\textit{target}).
Despite the two approaches are equivalent, in this paper we are going to focus on the former representation, as the latter provide to be quite inefficient for performing conformance checking of data-aware models \cite{bpm21}. Such declarative temporal rules are not limited to the mere presence of specific events within the trace, but also determine temporal occurrence patterns. Such models are usually exploited as the core behind an AI's temporal decision making. Therefore, a model defines an AI's unique behaviour, and by varying the set of expressions within these models, a entirely new set of responses can be generated.  Mining a hospital log consisting of successful traces (correct procedures, adequate medication etc.) could produce a model representing the patterns leading to `good' outcomes  \cite{Amantea2020}. Such mining approach could either retrieve either declarative models or sequential patterns of interest \cite{mining}. An AI could then query the resulting mined model for determining which clinical situation is likely to adhere to the expected clinical standards.

When a trace does not adhere to the model, we say that the trace is \textit{deviant} \cite{bpm21}. Given a Declare template \textsf{RespondedExistence}, \DeclareClause{RespondedExistence}{A}{\textbf{true}}{B}{\textbf{true}} is the instantiated Declare clause for event labels \texttt{A} and \texttt{B} that states \emph{`If event A happens, event B must happen sometime in the future'}: where A (B) is the \textit{activation} (\textit{target}) condition; a trace would be a deviant \emph{iff.} a trace contained an instance of A that was not followed by an instance of B. Furthermore, \textbf{true} predicates associated to activation (target) conditions can be enriched to express data conditions (\S\ref{sec:DAD}). As any declarative language, the specification of such templates can be expressed as a set of logical operators: in this field, Finite Liniear Time Logic (\LTLf) is usually exploited: for instance $\DeclareClause{RespondedExistence}{A}{$p$}{B}{$q$}$ becomes $\Future(A\wedge p) \Rightarrow \Future(B\wedge q)$, where $\Future(A\wedge p)$ express the intention that a trace labelled as $A$ satisfying a data condition $p$ should occur any time within the trace (\textit{ibid}.)

%%{\color{red}[TODO: replace] To further decompose these clauses, algebraic notation can be used to represent the set of operators that are the constituents of a given clause. Declare templates can be represented using \LTLf \cite{Li2020}. This allows for a flexible conformance checking implementation, as each clause can be represented as a unique pairing of \LTLf operators and join operators, for insatnce $\mathbf{RespondedExistence(A,B)}$ becomes $\Future(A) \Rightarrow \Future(B)$\PopUpComment{Giacomo}{Please observe that mathcal should be used only by surrounding the symbol of interest. Otherwise, in some other scenarios, you might have faults. Still, we are going to use a box/diamond notation for this paper. I added some macros for that}. As part of the process mining pipeline, conformance checking is used to identify patterns emerging from a given log. Therefore, process mining can actually be reduced to a conformance checking problem.}\PopUpComment{Giacomo}{Some of the contents in here are good, and should be put elsewhere in the introduction. "In fact, despite these operators might be applied to query plans similarly to relational algebra operators, no work -- to the best of our knowledge -- exploited this possibility". But, this should be linked to another kind of problem, too!}
\medskip

\paragraph*{Who might be interested in our solution? How these people might use this work?} \textit{Please provide the pieces of information that are specific to your own research field, and provide some use case examples motivating the practicality of your approach}  

Conformance checking can be applied to several different domains. First, given that process model information can be exploited to represent the tasks performed by both physical and cybernetic agents \cite{Ioanna}, this information can be exploited to detect \textbf{cyber-security attacks}, where a model can be extracted from previous invasions allowing common patterns to be identified \cite{BENASHER201551,LagraaS20}. Then, the conformity of any trace to the model might be exploited for determining whether an attack occurred or not. Second, temporal models extracted from hospital logs, consisting of diagnoses and treatments with their respective outcomes, could aid \textbf{healthcare} professionals in \textbf{decision making} \cite{Amantea2020}. In fact, such models allow to associate a precondition to a consequence within a specific clinical event of interest, thus providing whitebox explainable AI \cite{mining,KusumaKMHGJ20}. As a result of the process discovery, conformance checking tools might be exploited to assess whether the specific clinical case abides to the declarative rules associated to the model, thus allowing the prediction of a specific clinical event of interest. Last, most recent videogames also exploit AI features: existing state of the art already exploit automata \cite{Miyake2017} for modelling \textsc{Non Player Character}'s behaviours. As Declarative models and automata are completely equivalent approaches, developers might exploit the former to more compactly represent the latter. Furthermore, as debugging AI in videogames is a crucial challenge \cite{john2019debugging}, conformance checking solutions might be exploited for debugging unexpected behaviours in videogames. Furthermore, as AAA videogames already allow to track and log both players and NPC actions\footnote{\url{https://battlefieldtracker.com/}}, it might be also possible to use game logs for distinguishing winning strategies from losing ones \cite{mining}. As a result, analysis of an ongoing trace at runtime would then allow the model to `suggest' an action that is beneficial to the player based on the current state of the game, with the current strategy they are pursuing.

%\begin{itemize}
	%\item A cyber-security attack, where a model can be extracted from previous invasions allowing common patterns to be identified. Some solutions such as \cite{BENASHER201551} use technologies to identify this, but they do not use conformance checking. 
	%\item Suggesting actions to players in video games. Conformance checking applications for AI in video games has not previously been research; existing state of the art \cite{Miyake2017} use either automata or machine learning. Process mining would allow models to be extracted that represent unique strategies players have attempted in the past. Information regarding their `success' can also be stored. As a result, analysis of an ongoing trace at runtime would then allow the model to `suggest' an action that is beneficial to the player based on the current state of the game, with the current strategy they are pursuing.
%\end{itemize}




\paragraph*{What do I want to say (to the research community), precisely.} \textit{I want to communicate the general problem that I am aiming to solve} 
Current state of the art conformance checking solutions do not exploit the benefits of storing data in a custom relational database. When running queries, the same data is often accessed multiple times \cite{BurattinMS16,bpm21}. This is especially the case in the process of data-mining with large workloads \cite{SchonigRCJM16}, where the identification of patterns often share similar subqueries. On the other hand, existing solutions \cite{BellatrecheKB21} identify common sub-expressions within several queries running contemporarily, therefore reduce both the data access and the computation overhead to a minimum. This can be easily related to the conformance checking problem, where multiple declarative clauses from the same model might be assessed contemporarily. By decomposing these queries into \LTLf, a similar approach can also be followed. We propose that the queries, decomposed into \LTLf operators, can also follow a query plan similar to \cite{BellatrecheKB21}. We extend the approach by adapting the query plan to express \LTLf operators similarly to relational algebra operations, where common sub-expressions can still be rationalised. Still, there is some prior work on decomposing clauses into traditional SQL queries \cite{SchonigRCJM16}. This solution \emph{does} exploits the benefits of using a relational database (and therefore query plans) by transforming declare clauses into traditional SQL queries. However, this solution is limited as it neither considers data conditions (only event identifiers), nor considers multiple clauses pertaining to disparate Declare templates. This provides less functionality than we propose, where we are data-aware and theta conditions can be taken into account when performing any operators.

\paragraph*{Why do I want to talk about this problem? Why is it relevant?} \textit{Because current literature is lacking of a given aspect} 

%Conformance checking is an integral part of artificial intelligence that bridges data mining and business process management. 


Conformance checking can be extremely computationally intensive, both in time and storage, so optimised solutions are necessary to ensure a well performant implementation. To our knowledge, no solution existing whereby a relational base exploits optimised query plans, adapting solutions such as \cite{BellatrecheKB21}, in a business process environment using LTLf.
\medskip


\paragraph*{Now, communicate our idea also to the people working in our same area!} \textit{In particular, this means that we can go down in technicalities on what we want to solve, which are the primarily goals of our research, and which are the intermediate requirements/results leading to the results that we expect.} 

As process mining can be reduced to a conformance checking problem, a given log can be queried against a declarative model at runtime, and the same conformance checking calculations can be applied to generate its conformance \emph{at the current time}. Therefore, KnoBAB provides an optimized representation of the trace logs over which the declarative models $\mathcal{M}$ are going to be both queried and mined with LTLf.

We propose a knowledge base, KnoBAB, which provides efficient conformance checking by adapting query plan optimisations \cite{BellatrecheKB21} to LTLf. In addition, we provide data-aware capabilities, which discussed database solutions do not. KnoBAB provides the conformance of a \emph{trace} to a set of clauses, not the conformance of a clause against a log. This is more valuable in scenarios where trace information could point to where, and why, it was a deviant. Such knowledge could then be used for many features, such as generating the repair for this trace.
\medskip
 

The greatest amount of performance gain is due to the custom query plan, structured in such a way that multiple queries are stored within a graph, and then batch jobs are run using \textbf{parallelisation}. When process mining, large numbers of queries are performed, therefore there will be many instances of duplicate data accessing, resulting in poor optimisation. In this approach, there is the guarantee that unique data elements are obtained and processed only once, while current state of the art process-mining approaches access data per query. 

\section{Related Work}
%\textit{In this space, you usually want to quote the papers that you know that are near to the area. You want to make comparison, make similarities, and state which are their deficiencies if any} Questions that a reviewer might ask:
%\begin{itemize}
%	\item Is the current literature being assessed exhaustive, or is there something missing?
%	\item If the current literature us exhaustive, is it clear why each component is introduced?
%	\item For each piece of literature, are the different approaches compared and analysed, in weak and strong links? Which are the connections to your proposed approach?
%	\item E.g., are the opponents introduced as baselines for the experiment section also introduced in here?
%\end{itemize}


\paragraph*{XES Log Model}\label{sec:XES}


(Data) \textit{payloads} are maps  associating attributes (i.e., \textit{keys}) to data values. 
Given a finite set of activity labels $\textsf{Act}$, an event $\sigma_j^{i}$ is a pair $\Braket{\textsf{a},p}$, where $\textsf{a}\in\textsf{Act}$ is an activity label, and $p$ is a payload, mapping each key to a single value. 
A \textit{trace} $\sigma^i$ is a temporally-ordered and finite sequence of distinct events $\sigma^i=\sigma_1^i\cdots\sigma_n^i$, modelling a process run. 
All events within the same trace associate the same values to the same trace keys. 
A log $\mathcal{L}$ is a finite set of traces $\Set{\sigma^1,\dots,\sigma^m}$. We denote  $\Sigma\subseteq\textsf{Act}$ as the set of all the distinct activity labels in the log. If a payload is also associated to the whole trace, then this can be easily mimicked by adding an extra event containing such a payload, \textsf{\_\_trace\_payload}, at the beginning of the trace. {This is evidenced from \tablename~\ref{table:dataset}, where the \textsf{BPIC 2012} dataset contains by default 24 unique event labels, but after injecting the \textsf{\_\_trace\_payload} event, this increase to 25. } This  characterization is compliant with the \textsc{eXtensible Event Stream} (XES) format, which is the \textit{de facto} standard for  event logs %within the Business Process Management community 
\cite{XES}. 
\begin{table}
	\centering
\caption{Declare Templates, where $A\wedge p$ ($B\wedge q$) represents the activation (target) condition, where $A$ ($B$) denote the activity label condition, and $p$ ($q$) refers to the data payload condition.}\label{tab:dt}
\resizebox{\textwidth}{!}{\begin{tabular}{c|l|p{9cm}|l}
	\toprule
	Type & Name & Natural Language Specification for Traces & \LTLf Operators\\
	\midrule
	 \parbox[t]{2mm}{\multirow{4}{*}{\rotatebox[origin=c]{90}{\textit{Simple}}}} & \textsf{Init($A,p$)} & The trace should start with an activation & $A\wedge p$\\
	 & \textsf{Exists($A,p,n$)} & Activations should occur at least $n$ times & $\Future(A\wedge p \wedge \Next (\textsf{Exists} (n-1)))$\\
	 & \textsf{Absence($A,p,n+1$)}  & Activations should occur at most $n$ times & $\neg \textsf{Exists}$(n+1)\\
	 & \textsf{Precedence($A,p,B,q$)}  & Events preceding the activations should not satisfy the target & $\WeakUntil{\neg(B\wedge p)}{(A\wedge p)}$\\
	 \midrule
	 \parbox[t]{2mm}{\multirow{12}{*}{\rotatebox[origin=c]{90}{\textit{(Mutual) Correlation}}}} 	 & \textsf{ChainPrecedence($A,p,B,q$) }  & The target is immediately preceded by the activation. & $\Globally(\Next(B\wedge q)\Rightarrow (A\wedge p))$\\
	& \textsf{Choice($A,p,B,q$) }  & Either the activation or the target condition must appear. & $\Future(A\wedge p)\vee\Future(B\wedge q)$ \\
	 & \textsf{Response($A,p,B,q$) } & The activation is either followed by or simultaneous to  the target. & $\Globally((A\wedge p)\Rightarrow\Future(B\wedge q))$ \\
	 & \textsf{ChainResponse($A,p,B,q$) }  & The activation is immediately followed by the target. & $\Globally((A\wedge p)\Rightarrow \Next(B\wedge q))$\\
	 & \textsf{RespExistence($A,p,B,q$) }  & The activation requires the existence of the target.& $\Future(A\wedge p)\Rightarrow\Future(B\wedge q)$ \\
	 & \textsf{ExlChoice($A,p,B,q$) } & Either the activation \texttt{xor} the target happen. & $(\Future(A\wedge p)\vee \Future(B\wedge q))\wedge \DeclareClause{NotCoExistence}{A}{p}{B}{q}$\\ 
	 & \textsf{CoExistence($A,p,B,q$) }  & \textsf{RespExistence}, or vice versa. & $ \DeclareClause{RespExistence}{A}{p}{B}{q}\vee \DeclareClause{RespExistence}{B}{q}{A}{p}$\\
	 & \textsf{Succession($A,p,B,q$) }  & The target should only follow the activation. & $\DeclareClause{Precedence}{A}{p}{B}{q}\wedge \DeclareClause{Response}{A}{p}{B}{q}$\\

	 & \textsf{ChainSuccession($A,p,B,q$) }  & Activation immediately follows the target, and the target immediately preceeds the activation. & $\Globally((A\wedge p)\Leftrightarrow\Next(B\wedge q))$\\
	 & \textsf{AltResponse($A,p,B,q$) }  & If an activation occurs, no other activations must happen until the target occurs.  & $\Globally((A\wedge p)\Rightarrow(\DUntil{\neg(A\wedge p)}{(B\wedge q)}))$\\
	 & \textsf{AltPrecedence($A,p,B,q$) }  & Every activation must be preceded by an target, without any other
	 activation in between &   $\DeclareClause{Precedence}{B}{q}{A}{p}\wedge \Globally((A\wedge p)\Rightarrow \Next(\WeakUntil{\neg(A\wedge p)}{(B\wedge q)})$\\
	 \midrule
	 
	 \parbox[t]{2mm}{\multirow{2}{*}{\rotatebox[origin=c]{90}{\textit{Neg.}}}} & \textsf{NotCoExistence($A,p,B,q$) } & The activation \texttt{nand} the target happen.&  $\neg(\Future(A\wedge p)\wedge\Future(B\wedge q))$\\
	 & \textsf{NegSuccession($A,p,B,q$)} & The activation requires that no target condition should follow.& $\Globally((A\wedge p)\Rightarrow \neg\Future(B\wedge q))$ \\
	 \bottomrule
\end{tabular}}
\end{table}



\paragraph*{(Data-Aware) Declare and \LTLf}\label{sec:DAD} Temporal declarative languages model highly variable scenarios, where state machines provide complicated graph models that can be hardly understandable by the common business stake-holder \cite{PichlerWZPMR11}. Each single temporal condition is expressed through \textit{templates} (i.e., an abstract parameterized property: Table \ref{tab:dt} column 2), which are then instantiated on a set of real activation, target, or correlation conditions. We can then categorize each declare template from \cite{Li2020} by means of these conditions and the ability of expressing correlations between two temporally distant events happening in one same timeline (\textit{trace}): simple
 templates (Table \ref{tab:dt}, rows 1-3) only involving activation conditions; (mutual)
 correlation templates (rows from 4 to 15), which describe a dependency between two
activation and target conditions, thus including correlations between the two; and negative relation templates (last 2 rows), which describe a negative
dependency between two events in correlation. %Please observe that, 
Despite %some of 
these templates may appear quite similar, they generate completely different finite state machines, thus suggesting that these conditions are not interchangeable\footnote{\url{http://ltlf2dfa.diag.uniroma1.it/}}. 

As a formal basis for specifying temporal patterns, Declare adopts %the customary choice of of 
Linear Temporal Logic over finite traces (\LTLf), which interprets formulae over an onbouded, yet finite linear sequence of states. %In the context of this paper, consistently with the literature on business
% process execution traces, we make the simplifying assumption that in each point of the sequence, one and only one
%element from $\Sigma$ holds. 
Given a trace $\sigma^i$, the evaluation of a formula $\varphi$ is done in a given state (i.e., event id, or position) of the trace, and we use the notation $\sigma^i_j\vDash\varphi$ to express that $\varphi$ holds starting from the $j$-th event of the $i$-th trace. We also use $\sigma^i\vDash\varphi$ as a shortcut notation for $\sigma^i_0\vDash\varphi$. This
 denotes that $\varphi$ holds over the entire trace $\sigma$ starting from the very beginning. %and, consequently, logically captures the
 %notion of conformance of $\sigma$ against $\varphi$. 
 We say that $\varphi$ is \textit{satisfiable} if it admits at least one conforming trace. An \LTLf formula $\varphi$ is built by extending propositional logic with temporal operators in bold: \[\varphi:=\textsf{A}\wedge p\;|\;\neg \varphi\;|\;\varphi\vee \varphi'\;|\;\varphi\wedge\varphi'\;|\;\Next{\varphi}\;|\;\Globally{\varphi}\;|\;\Future{\varphi}\;|\;\DUntil{\varphi}{\varphi'}\] where $\Next{\varphi}$ denotes that the condition $\varphi$ should occur from the next state, $\Globally{\varphi}$ denotes that the condition has to hold on the entire subsequent path, $\Future{\varphi}$ denotes that the condition should occur somewhere on the subsequent path, and $\DUntil{\varphi}{\varphi'}$ denotes that $\varphi$ has to hold at least until $\varphi'$ becomes true, either at the current or a future state. Some operators can be seen as syntactic sugar: $\WeakUntil{\varphi}{\varphi'}:=\DUntil{\varphi}{\varphi'}\vee\Globally{\varphi}$, and $\varphi\Rightarrow\varphi':=(\neg \varphi)\vee (\varphi\wedge \varphi')$. Similarly to relational algebra, these operators also support equivalence rules, thus allowing to rewrite a given \LTLf expression in an equivalent one that might be more efficient to compute.

Despite this formulation has been already extended so to support correlation constraints \cite{BurattinMS16}, such a solution is affected by the following two deficiencies: first, correlation-conditions have to be represented alongside the target condition levels, thus hampering the exploitation of efficient relational database algorithms for correlation conditions via joins. Furthermore, these operators can only assess the validity of one trace at a time while, on the other hand, we might need to assess the satisfiability of multiple traces at the same time by composing partial results returned by each single operator. As a consequence, these operators cannot be directly exploited similarly to the relational algebra operators for relational databases. For this reasons, we propose a reformulation of such operators in \S\ref{ssec:xltlf} (\xLTLf). 

%\input{sections/21temporallogicoperators}
%\subsection{Data-Aware Conformance Checking}
%% Second paragraph: comparison with Burattin. 
\paragraph*{Data-Aware Conformance Checking}
%\begin{itemize}
%	\item{Briefly introduce what Burattin does}
%	\item{How this problem is related to ours?}
%	\item{How many clauses does Burattin consider at a time?}
%	\item{Still, can those be also different types of clauses or not?}
%	\item{On the other hand, do they have proper support for data conditions and event labels?}
%	\item{Do they use a Knowledge Base? If not, how are they querying?}
%	\item{What is their proposed parallelization approach? What is the difference and similarity with our approach?}
%\end{itemize}
%Previous solutions, while performing conformance checking, do not consider the addition of payload information for each event, limiting the expressiveness of the model. 
\texttt{Declare Analyzer}\footnote{\url{http://www.promtools.org/doku.php?id=prom611}} \cite{BurattinMS16} proposes one of the first solutions that provides conformance checking to data-aware logs. Declare templates are decomposed into \LTLf expressions, that not only contain event information, but a payload associated to each event per clause. This information manifests itself as \emph{activation} and \emph{correlation} conditions for to the \emph{activation} and \emph{target} events respectively. This approach is claimed to be is more optimised than the previous works \cite{VanDerAalst2005}.%, which do not even consider data conditions. 

While this solution exhibits good performance, it does not exploit the benefits that a relational database can provide, where techniques such as query optimisation can benefit computation. Due to this, they cannot capitalise on possible performance gains by running multiple queries so to minimize the data access as in \cite{BellatrecheKB21}. Another further gain that can be made from structuring queries in a query plan (that this approach does not support) is that it can be easily parallelised. In addition, it is inflexible; each clause is implemented from scratch, and, as they do not share the operators we provide, the addition of further clauses would require an entirely new implementation, while our solution support the definition of potential new Declare template via configuration files loaded in the initialization phase. We propose a more generalised solution, where each clause is composed of a combination of unique operators, allowing for any new clause to be included on the fly. 

While data conditions are considered, the proposed solution regards only one predicate per event (through target and correlation conditions). Therefore, multiple data conditions cannot be combined through either conjunction or disjunction, limiting the possible expressiveness of the conformance calculation. For example, queries such as 
$\DeclareClause{Response}{RainStart}{moistureContent\\>50\% \wedge consecutiveRain>3}{FloodCheck}{forecast=rain}$ 
%$\mathbf{Response(RainStart \{moistureContent>50\%  \& consecutiveRain>3\}, FloodCheck\{forecast=rain\})}$
are only possible by running a query per data condition and intersecting the returned set. KnoBAB provides the ability to combine several conditions per event, allowing queries like above to be processed. Furthermore, authors do not exploit efficient relational algebra operators when possible, as full-outer-theta-joins (or theta-joins) for unions (or conjunctions) with correlation conditions.


%\subsection{Process Mining through Conformance Checking}
\paragraph*{Process Mining through Conformance Checking}
%\begin{itemize}
%	\item{Briefly introduce what SQLMiner does}
%	\item{How this problem is related to ours}
%	\item{How many clauses does SQLMiner consider at a time?}
%	\item{Still, can those be also different types of clauses or not?}
%	\item{On the other hand, can we potentially do it? does our solution, on the other hand, provide a constraint on the type of the model, or is this general enough to consider each possible clause at the same time?}
%\end{itemize}
Some approaches utilise conformance checking as a mechanism to mine declarative \RevRepl{processes}{models}  from an event log \RevAdd{by testing the validity of each possible clause over each possible trace}. SQLMiner \cite{SchonigRCJM16} \RevRepl{uses traditional}{does so via} SQL queries \cite{Schonig15} \RevDel{as a to exercise this}\RevRepl{, E}{where e}ach specified declarative \RevRepl{constraint}{Template}, e.g. \emph{Response}, \RevRepl{has been}{is} converted into a SQL representation. \RevAdd{Then, the outcome of such a query is a declare clause distinct by different activation and target combination.} 

To achieve this, the event log is \RevAdd{first} loaded into the database as table \RevDel{, on which the queries are actioned}. \RevRepl{F}{Then, f}or every activation and target combination \RevRepl{a parametrized template of the the current clause is generated, and its conformance against the table}{of the declarative template, the authors store the candidate activation and target conditions to be tested in another relational table}. \RevRepl{To extend the functionality of each clause,}{For scoring the validity of each candidate clause}, the authors also calculate \emph{Support} and \emph{Confidence} values, to determine the precision and reliability of the calculation respectively. Records which do not pass pre-determined \emph{Support} and \emph{Confidence} thresholds are filtered from the data. \RevAdd{Despite } SQL also supports data constraints, \RevAdd{this process considers neither activation, nor target, nor correlation conditions}.

\RevAdd{Despite the authors } provide additional \RevAdd{data} perspectives through `Resource Assignment Constraints' clauses, distinct from the Declare ones, these additions were only considered as an additional perspective \RevRepl{as a whole}{to the global trace payload}, and do provide the functionality necessary for payload on an event\RevDel{, nor for Declare}. Furthermore, this data exists in isolation from the event information, and is only used as a filter on the data in the log table. KnoBAB provides payload information \RevAdd{both \emph{per trace}} and \emph{per event}, which could also be stored in a separate table as SQLMiner suggests, thus providing greater expressiveness per clause.

SQLMiner queries can be chained together, using \texttt{\textbf{SET UNION}}, though this provides no \RevAdd{possibility for testing which are the clauses that are satisfied by the majority of the traces.}, Furthermore, these query plans are not optimized as in \cite{BellatrecheKB21}, thus failing at minimizing the data access. This is inferior to KnoBAB, which has the ability to process multiple queries simultaneously. This allows for further optimisation of the query plan, which can exploit common sub-expression \emph{across queries}. This query plan can then be parallelized and processed with batch computation per query layer.

\section{Logical Model}
\subsection{XES Log Model}\label{sec:XES}

\texttt{\color{red}[TODO: simplify, remove unnecessary logical notation] } (Data) \textit{payloads} are maps  associating attributes (i.e., \textit{keys}) to data values. %We consider also the case in which the value of a certain key $k$ is missing in a payload. In particular, we denote as $\varepsilon$ an element $\varepsilon\notin V$, such that $p(k)=\varepsilon$ for $k\notin\textup{dom}(p)$. 
Given a finite set of activity labels $\textsf{Act}$, an event $\sigma_j^{i}$ is a pair $\Braket{\textsf{a},p}$, where $\textsf{a}\in\textsf{Act}$ is an activity label, and $p$ is a payload. %; we denote with $\lambda$ (and $\varsigma$) the first (and second) projection of such pair, i.e., $\lambda(\sigma_j^{i})=\texttt{A}$ (and $\varsigma(\sigma_j^{i})=p$). 
A \textit{trace} $\sigma^i$ is a temporally-ordered and finite sequence of distinct events $\sigma^i=\sigma_1^i\cdots\sigma_n^i$, modeling a process run. %We distinguish the trace keys ($K_t$) from the event keys ($K_e$), such that $K=K_t\cup K_e$ with $K_t\cap K_e=\emptyset$: 
all events within the same trace associate the same values to the same trace keys. %, i.e., $\forall \Braket{\texttt{A}_i,p_i},\Braket{\texttt{A}_j,p_j}\in\sigma.\;\forall k\in K_t.\; p_i(k)=p_j(k)$. 
A log $\mathcal{L}$ is a finite set of traces $\Set{\sigma^1,\dots,\sigma^m}$. We denote  $\Sigma\subseteq\textsf{Act}$ as the set of all the possible activity labels in the log. If traces also contain a payload, then this can be easily mimicked by adding an extra event containing such a payload, \textsf{\_\_trace\_payload}, at the beginning of the trace. This  characterization \cite{bpm21} is compliant with the \textsc{eXtensible Event Stream} (XES) format, which is the \textit{de facto} standard for representing event logs within the Business Process Management community \cite{XES}. Please refer to \S\ref{ssec:dl} for the representation of each log as a separated in-memory database.

\subsection{Result Representation}
\texttt{\color{red}[TODO: each event in the data input can be represented as a single record, which consists of...] That is the Logical Model representation of the data}

\section{\xLTLf operators}\label{sec:xltlf}
Algoritms \ref{algo:xltlfAlgo}. \texttt{\color{red}[TODO: \xLTLf Operators]} 

\begin{mdframed}[hidealllines=true,backgroundcolor=orange!20]
	{$\color{red}\ostar$} On the other hand, \xLTLf operators are associating both or and and operators to such operators, thus allowing the implementation of the and (or) operator as a (full-outer) join algorithms. By doing so, we can also associate the correlation condition to implication operators after rewriting $\varphi\Rightarrow_\Theta\varphi'$ as $(\neg\varphi)\vee(\varphi\wedge_\Theta\varphi')$.
\end{mdframed}

\begin{itemize}
	\item What is the input and output format for all of the operators
	\item What does the sorting assumption guarantee at the operators' implementation level
	\item Discuss the data operators at the leaf level for accessing data, and how do we guarantee to preserve such representation.
	\item Describe the intuition over each operator
\end{itemize}
\section{KnoBAB Architecture}
The methodology behind its design has been an abstract one, with the only bespoke characteristics being its ability to generate conformances efficiently. \texttt{\color{red}[TODO: some gluing paragraph, if enough space is left]}


\subsection{Data Loading}\label{ssec:dl}
The data loading phase   loads logs  serialized in multiple  formats, thus uncluding the XML-based XES standard, a tab-separated events' activity labels, and the \textsc{Human Readable Log Format} firstly introduced in \cite{bpm21}. We use different data parsers, which are still linked to the same data loading primitives. 

If the log does not contain data payloads, the entire log can be represented into two relational tables, \textsf{CountingTable(ActId,TraceId,Count)} and \textsf{ActivityTable(ActId,TraceId,EventId,\\PrevEvent,NextEvent)}. While the former counts the occurrence of each activation label in $\Sigma$ for each trace, the latter lists all of the possible events similarly to SQLMiner. Both tables compactly represent the initial three columns as a 64-bit unsigned integer, which is also used to sort the tables in ascending order. A row $\braket{\textsf{a},j,h}$ from \textsf{CountingTable} states that there are $h$ events exhibiting the activity label $\textsf{a}$ in the trace $\sigma^j$; each row $\braket{\textsf{a},j,i,q,q'}$ from \textsf{ActivityTable} states that the $i$-th event of the $j$-th trace ($\sigma^j_i=\braket{\textsf{a},p}$) is labelled as $\textsf{a}$, while $q$ (or $q'$) is the pointer to the immediately preceding $\sigma^j_{i-1}$ (or  following, $\sigma^j_{i+1}$) event within the trace, if any. As seen from \figurename~\ref{fig:knobab_pipeline}, the selected block for the \textsf{ActivityTable} highlights this structure, where NULL for a given row regards  the start (finish) event of each trace, where there is no possible reference to past (future) events. Trace payload information is injected (as an event) before the first event, which is also contained. Therefore, all trace payload events contain NULL in all of their previous fields.


If, on the other hand, the log is associated to either trace or event payloads, we exploit 
the query and memory efficient  column-based model \cite{IdreosGNMMK12}, thus representing all of the values $v$ associated to a  payload key $k$ within the rows from  \textsf{AttributeTable$k$}. In our implementation, each row $\braket{\textsf{a},v,i}$ from  \textsf{AttributeTable$k$(ActId,Value,Offset)} represents a value $v$ associated to the key $k$ from an event stored as the $i$-th row from \texttt{ActivityTable}. To perform payload-based queries efficiently, the table is sorted in ascending order by the  three columns. As each data condition is always associated to a given activity label, those can be effectively run as data range queries run via binary search algorithms. From \figurename~\ref{fig:knobab_pipeline}, all the attributes are stored in distinct tables. The value column can contain multiple data types, but each attribute is associated to only one type. When decomposed atoms are used for a query, the tables associated to the query are then accessed. The offset value can then determine the location within the \textsf{ActivityTable} where this event occurred, providing the trace id and event id required for the intermediate representation.

\textsf{CountingTable} is mainly accessed for existential and \textsf{Exists} and \textsf{Absence} templates where no data payload is specified, while  \textsf{ActivityTable} is  used for either returning all of the events within the log associated to a given activity label, or returning all of the events happening at either the beginning or at the end of a trace. Each table \textsf{AttributeTable$k$}, on the other hand, will %accessed for 
return all the events satisfying a given condition associated to a specific %data 
key $k$. %As each data condition is always associated to a given activity label, each value is grouped by activity label via table ordering.

After loading the whole dataset, the number of the traces within the log ($|\mathcal{L}|$), the length $\ell_j$ for each trace $\sigma^j$, and the number of distinct activity labels ($|\Sigma|$) is known. Given this, we can get the number of occurrences of each $i$-th activity label from $\Sigma$ in each trace by directly accessing the rows within the \textsf{CountingTable} within the range $[|\mathcal{L}|\cdot (i-1) + 1,\; |\mathcal{L}|\cdot i]$.
%we can determine the number of occurrences of each $i$-th activity label in $\Sigma$ for each trace $\sigma^j\in\mathcal{L}$ by accessing the rows from  without the need of any external indexing data structure. 
On the other hand, the loading and and indexing phase generates an \textsf{ActivityTable} associated to two indices, a primary index and a secondary index. While the former allows to effectively return all of the events associated to a specific activity label, the latter is used to access to either the first and to the last event in a trace. If required, pointers associated to each record allow temporally scanning the traces as a double linked list. From \figurename~\ref{fig:knobab_pipeline}, the offset for the third event (B) in the \textsf{CountingTable} can then be calculated by would be $[3 \cdot (3-1) + 1, 3 \cdot 3] = [7,9]$. Given that this counting table computes only for untimed operations, the intermediate result for untimed \textsf{Exists}$(1,B,\textbf{true})$ within this range are provided as $\Set{\braket{1,1,\emptyset}, \braket{2,1,\emptyset}}$, as the third trace never contains a B, therefore is not included in the result.
Details of the loading and indexing phase are omitted due to the page limits.




\subsection{Query Compiler}\label{sec:qc}
The query compiler is structured into three main phases. \textit{(i)} The \textit{atomization pipeline}  rewrites the data predicates 
associated with each activity label as a 
disjunction of mutually exclusive data conditions. We can tune KnoBAB to always atomize each possible activity label if it exists any Declare Constraint associating it to a data condition as in \cite{bpm21}, or we can choose to provide such an interval decomposition only to the Declare constraints exhibiting data conditions. While the former approach will maximise the access to the \textsf{AttributeTable}s, the latter will maximise the access to the \textsf{ActTable}. By doing so, we can ensure that the data satisfying some given properties can be visited at most once, thus guaranteeing the assumptions from \cite{BellatrecheKB21} also at the data accessing level. Correlation conditions do not undergo this rewriting step. The atomized model in \figurename~\ref{fig:knobab_pipeline} replaces the non-correlation data predicates with the outcome of the atomization process as in \cite{bpm21}. 


We \textit{(ii)} rewrite each Declare constraint as a \xLTLf formula, where the activations (and the potential target) conditions are instantiated with either just activity labels or also with associated data conditions as per the previous atomization step. 
Each sub-expression appears at most once as in \cite{BellatrecheKB21} by representing every single node in the query plan at most once: this is ensured by an internal query manager cache. The resulting query plan considering the simultaneous execution of multiple queries can be represented as a \textsc{Direct Acyclic Graph} (DAG).  
For each declarative clause appearing more than once (e.g., $m>1$), the associated \xLTLf expression will be computed at most  once, while its resulting data is going to be accessed $m$ times by the final aggregator: as per \figurename~\ref{fig:knobab_pipeline}, despite \textsc{Response} might be considered a subquery of \textsc{Succession}, the Max-SAT is still going to retrieve the output provided by the associated sub-expression. Green arrows remark operators' output shared among operators. Please also observe that operators with the same name and arguments but marked either with activation, target, or no specification are considered different as they provide different results, and therefore are not merged together. 
%\textit{Second}, we rewrite each Declare constraint as a \xLTLf formula, where the activations (and the potential target) conditions are instantiated with either just activity labels or also with associated data conditions as per previous atomization step. This step is mediated through a configuration file loaded at warm-up time, where novel declare constraints can be expressed as \xLTLf formulae. We guarantee to represent each subquery at most once as in \cite{BellatrecheKB21} by representing each node in the query plan at most once: this is ensured by an internal cache. At the end of this process, the query plan considering the simultaneous execution of multiple queries can be represented as a \textsc{Direct Acyclic Graph} (DAG). 
%>>>>>>> main
%>>>>>>> b69bdd98c4f898e5f4801fc44cbae460ecd04aef
This includes distinctions between timed and untimed operators.

Given that our execution engine provides the possibility of running a query plan in either a parallel or a sequential mode, we need   an additional step. \textit{(iii)} The previous DAG  represents a dependency graph, where a link between an ancestor and one of its descendants implies that the latter has to be computed before the former, thus suggesting an execution order. \figurename~\ref{fig:knobab_pipeline} depicts this as an arrow starting from the ancestor. To enforce that, we perform a lexicographical order over the DAG, through which we compute the maximum depth level associated with each node of the graph. %After doing so, 
We then represent the query graph as a stack of depth levels, where each operator on it can be run in parallel alongside its siblings.
 This proves that the computation of Declare Clauses can be reduced into an embarrassingly parallel problem, as the layered execution guarantees that no thread communication needs to happen, and that multiple threads could access contemporary the partial results associated with the immediately-descendant operators, as the former will return all of the events where the condition happened, while the latter will just return the trace event satisfying such condition alongside the required activations/targets listed in $L$. Furthermore, the proposed parallelization ensures minimizing the data access for computing the query. The DAG \figurename~\ref{fig:knobab_pipeline} depicts a query plan.

\subsection{In-Memory Storage Manager}

\subsection{Execution Engine} \label{ssec:xltlf}
At the time of the writing, KnoBAB supports four different types of model aggregation queries: Conjunctive Query, Max-SAT, \textsc{Confidence}, and \textsc{Support}. As we will see at the end of the subsection, these will not require a change on the query plan, but just a different way to integrate the intermediate representation $\phi_i$ returned by each declarative clause $c_i$. 


\textit{First}, the execution engine takes both the relational database resulting from the data loading and the DAG returned by the query compiler, and uses the leaf nodes from the latter to access the former. By query plan construction, all of the relevant data parts are going to be accessed at most once and then transformed into the expected intermediate result representation. \textit{Second}, the intermediate results are propagated from the leaves towards each root node associated with a declarative clause $c_k$. Any intermediate representation is always associated with each operator returning it as a temporary primary-memory cache. Each intermediate cache  might be completely freed if we are not computing a  \textsc{Confidence} query and if the furthest ancestor has already accessed it, or if it is a cache non-associated to an activation required by \textsc{Confidence} and the furthest ancestor has already accessed it. \textit{Third}, when the computation will finish running the shallowest DAG depth level containing the \xLTLf root associated with the entry-point of each declarative clause $c_k$, each of these operators will have an intermediate result $\phi_k$ stating all the traces satisfying $c_k$.

The {Conjunctive Query} will return the traces satisfying all of the Declare clauses via the intersection of all of the clauses via \textsf{And} and \textbf{true} as a $\Theta$ condition. Max-SAT will count, for each log trace $\sigma_i$, the intermediate results $\phi_k$ associated with each clause $c_k$ containing it, and then provide the ratio of such value over the total number of the model clause $|\mathcal{M}|$. By denoting as $\textsf{ActLeaves}(\phi_k)$ the untimed union of the intermediate results returned by the activation conditions for the declare clause $c_k\in\mathcal{M}$, the \textsc{Confidence} for $c_k$ is the ratio between the total number of traces returned by $\phi_k$ and the overall traces containing an activation condition. Dividing the total number of traces returned by $\phi_k$ by the total log traces returns the \textsc{Support}. Once each $\phi_k$ per clause $c_k$ is computed, the aggregation functions can be then expressed as follows:
\[\textup{ConjQuery}(\phi_1,\dots,\phi_n)=\textsf{And}(\phi_1,\dots\textsf{And}(\phi_{n-1},\phi_n,\textbf{true}),\textbf{true})\]
\[\textup{Max-SAT}(\phi_1,\dots,\phi_n)=\left(\frac{|\Set{k|\exists j,L. \braket{i,j,L}\in\phi_k}|}{|\mathcal{M}|}\right)_{\sigma^i\in \mathcal{L}}\]
\[\textsc{Confidence}(\phi_1,\dots,\phi_n)=\left(\frac{|\Set{i|\exists j,L. \braket{i,j,L}\in \phi_k}|}{|\textsf{ActLeaves}(\phi_k)|}\right)_{c_k\in \mathcal{M}}\]
\[\textsc{Support}(\phi_1,\dots,\phi_n)=\left(\frac{|\Set{i|\exists j,L. \braket{i,j,L}\in \phi_k}|}{|\mathcal{L}|}\right)_{c_k\in \mathcal{M}}\]
As the user in \figurename~\ref{fig:knobab_pipeline} asks the ratio between satisfied clauses over the model size, the query plan exhibits a Max-SAT aggregation.



\begin{algorithm}
	\caption{LTL$_f$ pseudocode implementation for the basic timed operators}


\begingroup % trick algorithm2e into thinking we're in one column mode
\csname @twocolumnfalse\endcsname
\noindent
\resizebox{\textwidth}{!}{%
	\begin{minipage}{1.3\textwidth}
%%%%%%%%%%%%%%
	\begin{algorithmic}[1]
		\Statex
\Function{Future}{$\varphi$}
		\State {\textbf{Require:\;}sorted($\varphi$)}
		\ForAll{$\braket{t,e,L}\in\varphi$}
		 \textbf{yield} $\braket{t,e-1,\bigcup\Set{L'|\braket{t,e',L'}\in\varphi\texttt{\textbf{\;and\;}}e'\geq e}}$
		\EndFor
\EndFunction\medskip
%%
\Function{Globally}{$\varphi$}
		\State {\textbf{Require:\;}sorted($\varphi$)}
		\ForAll{$\braket{t,e,L}\in\varphi$}
		\State $E\gets\Set{e'|\braket{t,e',L'}\in\varphi\texttt{\textbf{\;and\;}}e'\geq e}$
		\If {$|E|=\ell_t-e$} 
		 \textbf{yield} $\braket{t,e-1,\bigcup\Set{L'|\braket{t,e',L'}\in\varphi\texttt{\textbf{\;and\;}}e'\in E}}$
		 \EndIf
		\EndFor
\EndFunction\medskip
%%
\Function{Next}{$\varphi$}
		\State {\textbf{Require:\;}sorted($\varphi$)}
		\ForAll{$\braket{t,e,L}\in\varphi$\textbf{\;s.t.} $e>1$}
		\textbf{yield} $\braket{t,e-1,L}$
		\EndFor
\EndFunction\medskip
%%
\Function{CommonJoin}{$\varphi_1,\varphi_2,\Theta,{isDisjunctive}$} %\Comment{Sort Merge (Full-Outer-)Join}
	\State {\textbf{Require:\;}sorted($\varphi_1$)\textbf{\;and\;}sorted($\varphi_2$)}
	\State  $it\gets$\textbf{Iterator}$(\varphi_1), it\gets$\textbf{Iterator}$(\varphi_2)$
	\While{$it\neq\emptyset$\textbf{\;and\;}$it'\neq\emptyset$}
	\State $\braket{t,e,L}\gets\texttt{current}(it)$, $\braket{t',e',L'}\gets\texttt{current}(it')$
	\If{$t=t'$ \textbf{and} $e=e'$}
	\If{$L=\emptyset$} $L''\gets L'$
	\ElsIf{$L'=\emptyset$} $L''\gets L$
	\Else 
	\State $L''\gets\emptyset$
	\ForAll{$m\in L$\textbf{\;and\;}$n\in L'$\textbf{\;s.t.\;}$\Theta(m,n)$}
	\State $L''\gets L''\cup\Set{\textbf{join}[m,n]}$
	\EndFor
	\EndIf
	\State \textbf{yield} $\braket{t,e,L''}$; $\texttt{next}(it)$; $\texttt{next}(it')$; 
	\ElsIf{$t<t'$\textbf{\;or\;}($t=t'$\textbf{\;and\;}$e<e'$)} 
	\If  {$isDisjunctive$} \textbf{yield} $\braket{t,e,L}$  \EndIf
	\State $\texttt{next}(it)$
	\Else 
		\If  {$isDisjunctive$} \textbf{yield} $\braket{t',e',L'}$  \EndIf
		\State $\texttt{next}(it')$
	\EndIf
	\EndWhile
\EndFunction\medskip
%%
\Function{And}{$\varphi_1,\varphi_2,\Theta$}
	\Call{CommonJoin}{$\varphi_1,\varphi_2,\Theta,\textbf{false}$}
\EndFunction\medskip
%%
\Function{Or}{$\varphi_1,\varphi_2,\Theta$}
\Call{CommonJoin}{$\varphi_1,\varphi_2,\Theta,\textbf{true}$}
\EndFunction\medskip
%%
\Function{Until}{$\varphi_1,\varphi_2,\Theta$}
\State {\textbf{Require:\;}sorted($\varphi_1$)\textbf{\;and\;}sorted($\varphi_2$)}
\ForAll{$t$ \textbf{\;s.t.\;} $\braket{t,i',L'}\in \varphi_2$}
\State $\alpha\gets 0$; $Map\gets\{\}$
\ForAll{$\braket{t,i,L}\in\varphi_2$}
\While{$i<\alpha$}
\If{$\braket{t,\alpha,L_\alpha},\braket{t,\alpha+1,L_{\alpha+1}},\dots,\braket{t,i-1,L_{i-1}}\in\varphi_1,$\\\textbf{\;and\;}$\Theta(\alpha,j)$\textbf{\;for all\;}$j\in L_\alpha\cup\dots\cup L_{i-1}$}
\State $Map[i]\gets\Set{\texttt{\textbf{join}[k,i]}|k\in L_\alpha\cup\dots\cup L_{i-1}}$
 
\EndIf
\EndWhile
\State $Map[i]\gets Map[i]\cup L$; 
\ForAll{$i\in \texttt{mapKey}(Map)$} \textbf{yield} $\braket{t,i,Map[i]}$
\EndFor
\EndFor
\EndFor
\EndFunction\medskip
	\end{algorithmic}
%%%%%%%%
\end{minipage}%
}% <------------- end of \resizebox
\endgroup
\end{algorithm}
\subsection{LTLf Operators' implementation on the Physical Model}
\begin{table}
\caption{Queries over the BPI 2012 Challenge from \cite{BurattinMS16}}
\centering
	\resizebox{\textwidth}{!}{\begin{tabular}{l|c|c}
		\toprule
		\multirow{2}{*}{\textit{Query}} & \multicolumn{2}{c}{Query Time \textit{(ms)}} \\ 
		& \textsf{Declare Analyzer} & \textbf{KnoBAB}\\
		\midrule
		$q_1:= \DeclareClause{Response}{A\_SUBMITTED}{\textbf{true}}{A\_ACCEPTED}{\textbf{true}}$ &  $1.75\cdot 10^3$ & $1.83\cdot 10^1$\\
		$q_2:= q_1\wedge \MonoDeclareClause{Exists}{\_\_trace\_payload}{\texttt{AMOUNT\_REQ}\geq 10^3}{\geq 1}$ &  $1.77\cdot 10^3$ & $2.06\cdot 10^1$ \\
		$q_3:=q_1\wedge \MonoDeclareClause{Exists}{\_\_trace\_payload}{\texttt{AMOUNT\_REQ}< 10^3}{\geq 1}$ & $1.62\cdot 10^3$ & $2.15\cdot 10^1$\\
		$q_4:=q_1\textrm{ where }\texttt{A\_SUBMITTED.org:resource}=\texttt{A\_ACCEPTED.org:resource}$ & $1.71\cdot 10^3$ & $2.43\cdot 10^1$\\
		$q_5:=q_1\textrm{ where }\texttt{A\_SUBMITTED.org:resource}\neq\texttt{A\_ACCEPTED.org:resource}$ & $1.93\cdot 10^3$ & $2.59\cdot 10^1$\\
		$q_1\wedge q_2$ & $1.81\cdot 10^3$ &  $2.70\cdot 10^1$\\
		$q_1\wedge q_2\wedge q_4$ & $2.49\cdot 10^3$ & $2.74\cdot 10^1$ \\
		$q_1\wedge q_3\wedge q_4$ & $2.38\cdot 10^3$ & $2.51\cdot 10^1$\\
		$q_1\wedge q_2\wedge q_5$ &$2.06\cdot 10^3$  & $2.57\cdot 10^1$\\
		$q_1\wedge q_3\wedge q_5$ & $2.19\cdot 10^3$ & $2.29\cdot 10^1$\\
		$q_1\wedge q_2\wedge q_3\wedge q_4\wedge q_5$ & $2.90\cdot10^3$ & $2.66\cdot 10^1$ \\
		\bottomrule
	\end{tabular}}
\end{table}


\section{Experimental Analysis}
The benefits from the custom query plan are most obvious in the process mining stage, where a log consisting of potentially thousands of traces is tested against all combinations of clauses. However, computational gains can also be evidenced when the same querying approach is adapted to a runtime scenario, where we are querying only 1 trace against an existing model (which requires much less computation as a whole).

For $\mathcal{C}$ Declare clauses, where $\mathcal{N}$ is the data loading cost, implementations without a KB suffer, resulting in $\mathcal{O(C \cdot N)}$ complexity. With a KB, data loading is necessary only once, enhancing the complexity to $\mathcal{O(C + N)}$.

However these are computationally bottlenecked to the efficiency of these systems themselves, regardless of the optimality of the conformance checking.

SQL miner, due to the query structure, requires vast amounts of secondary memory for temporary caching of query computation, \highlight{much less than KnoBAB requires}.
\section{Conclusions and Future Works}
We propose KnoBAB, a fully relational database architecture for computing Conformance Checking via conjunctive queries, as well Max-SAT and clause \textsc{Confidence}/\textsc{Support} functions.  KnoBAB consists of a data loader and indexer, query compiler, and an execution engine, thus fully matching the architecture of a relational database. This solution was enabled by the extension of the traditional \LTLf operators, providing algebraic semantics to declarative temporal models, so as to support data operations over tuples representing trace  events. Our solution is not limited to one single declarative language of choice, as it might support any possible model that can be expressed via \xLTLf operators. Based on the latest solutions in current database literature, the query plan was also designed to minimize the data access by running the common sub-queries at most once.
%
KnoBAB outperforms state of the art solutions both tailored to the specific dataset or based on traditional relational databases running SQL queries.  This solution will enable us to learn models exploiting abductive reasoning rather than traditional mining techniques, thus also providing safety guarantees over noisy data and models that are inconsistency free \cite{PicadoDTL20}. 

Future works will provide extensive benchmarks for bigger log datasets and will provide speed-up results for the parallelized execution of the resulting query plan: despite this being already implemented, we postpone those results due to the lack of space in the present paper. For the time being, the logs available from the research community are quite compact, and therefore the whole dataset is well fit in primary memory. Dealing with actual big data solutions or bigger models will require us to migrate the data store location to secondary memory, thus requiring the adoption of Near-Data Processing techniques \cite{GuYBJLYKKYCJC16}. {As part of the data-loading phase, \textsc{Human Readable Log Format} key names currently only support strings consisting of letters. A proposed extension would allow for any possible string name, including numbers and symbols. } 

%Last, 
The adoption of relational databases and operator-based query plans might enable incremental trace updates so to extend those at runtime: this open research problem  can be now  solved by exploiting algebraic rewriting rules similar to the ones from relational databases, thus requiring a formal definition of \xLTLf operators. 

%\textit{Summarize the abstract even more, as now you need also to summarize the outcome of the experiments. Furthermore, say what was legitimately left out, and which are the future works being scheduled for extending the present work.}
%
%In particular:
%\begin{itemize}
%	\item Are all of the open questions in the introduction closed at this point? Are all the questions answered? 
%	\item If something relevant is left out, is that considered for future work?
%\end{itemize}
%
%Experiments on parallelizations are omitted due to lack of space. Are implemented, but they are going to be tested in our future work.
%
%Talk about the support of runtime traces (require much less computation than data mining - many fewer clauses, much less data (only 1 trace))
%
%Another possible addition could be the inclusion of external data conditions that are not bound to data payloads. We currently consider target, correlation and theta conditions, but not external factors. 
%
%Therefore, we can represent: 
%
%\emph{`If it has started raining, and the moisture content of the soil is above 50 percent, a flood check must happen if more rain is forecast'}.  
%As a Declare clause, this would formulate:
%
%{\tiny$\mathbf{Response(RainStart \{moistureContent>50\%\}, FloodCheck\{forecast=rain\}) WHERE RainStart\{location\} = FloodCheck\{location\}}$}
%
%What KnoBAB does not currently support, however, is the addition of an external global condition. 
%
%Therefore a constraint such as:
%
%\emph{`If it has started raining, and the moisture content of the soil is above 50 percent, a flood check must happen \textbf{2 days after} if more rain is forecast'}.  
%
%Is not currently supported. The inclusion of these conditions would allow for greater expressiveness of each clause, and therefore the declarative model itself.

%%
%% The acknowledgments section is defined using the "acks" environment
%% (and NOT an unnumbered section). This ensures the proper
%% identification of the section in the article metadata, and the
%% consistent spelling of the heading.
\begin{acks}
Samuel Appleby's work is supported by Newcastle University.
\end{acks}

%%
%% The next two lines define the bibliography style to be used, and
%% the bibliography file.
\bibliographystyle{ACM-Reference-Format}
\bibliography{deanon.bib,refs.bib}


%%
%% If your work has an appendix, this is the place to put it.
%\appendix

%\section{Research Methods}
%
%\subsection{Part One}
%
%Lorem ipsum dolor sit amet, consectetur adipiscing elit. Morbi
%malesuada, quam in pulvinar varius, metus nunc fermentum urna, id
%sollicitudin purus odio sit amet enim. Aliquam ullamcorper eu ipsum
%vel mollis. Curabitur quis dictum nisl. Phasellus vel semper risus, et
%lacinia dolor. Integer ultricies commodo sem nec semper.
%
%\subsection{Part Two}
%
%Etiam commodo feugiat nisl pulvinar pellentesque. Etiam auctor sodales
%ligula, non varius nibh pulvinar semper. Suspendisse nec lectus non
%ipsum convallis congue hendrerit vitae sapien. Donec at laoreet
%eros. Vivamus non purus placerat, scelerisque diam eu, cursus
%ante. Etiam aliquam tortor auctor efficitur mattis.
%
%\section{Online Resources}
%
%Nam id fermentum dui. Suspendisse sagittis tortor a nulla mollis, in
%pulvinar ex pretium. Sed interdum orci quis metus euismod, et sagittis
%enim maximus. Vestibulum gravida massa ut felis suscipit
%congue. Quisque mattis elit a risus ultrices commodo venenatis eget
%dui. Etiam sagittis eleifend elementum.
%
%Nam interdum magna at lectus dignissim, ac dignissim lorem
%rhoncus. Maecenas eu arcu ac neque placerat aliquam. Nunc pulvinar
%massa et mattis lacinia.

\end{document}
\endinput
%%
%% End of file `sample-sigconf.tex'.
