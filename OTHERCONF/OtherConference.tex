\documentclass[11pt]{article}
\setlength{\rightmargin}{2.5cm}
\setlength{\leftmargin}{2.5cm}
\usepackage{setspace}
\onehalfspacing
\textwidth 6.5in
\setlength{\topmargin}{0in}
\setlength{\headheight}{0in}
\setlength{\headsep}{0in}
\setlength{\textheight}{9in}
\setlength{\oddsidemargin}{0in}
\setlength{\textwidth}{6.2in}
\usepackage{xcolor}
\begin{document}
%\setlength{\rightmargin}{2.5cm}
%\setlength{\leftmargin}{2.5cm}
%\setleftmargin{2.5cm}	
%\setrightmargin{2.5cm}	
\title{Running Temporal Logical Queries on Knowledge Bases: a Business Process Management Perspective}

\author{[Authors Removed For Double Blind Review]}

\maketitle
~\\
\begin{abstract}
State of the art approaches on model checking for declarative temporal models exploit straightforward and computationally intensive solutions, where data needs to be scanned multiple times per temporal model (query). With a KB, data loading and indexing is necessary only once, thus enhancing the complexity and allowing queries to be run multiple times after loading the traces only once. In fact, no solution exploiting efficient relational databases, querying multiple declarative constraints, and expressing temporal correlation conditions as relational join operations is known. The resulting system can be then exploited in real-time scenarios, such as video games and cyber-security systems. For efficiently exploiting such a KB, this paper proposes for the first time, to the best of our knowledge, a specialised set of operators determining both the semantics of the temporal model and the operators for the query plan. As our trace logs are data-aware, we make also such operators support join conditions for the first time. Experiment shows\dots
\end{abstract}
~\\

{\bf Keywords}: TODO
~\\

% A category with the (minimum) three required fields
{\bf Topics} TODO
~\\


\section{Introduction}
Business Process Management (BPM) has been utilised in the development of models for data-aware logs.

\texttt{\color{red}[TODO:]}  For $\mathcal{C}$ Declare clauses, where $\mathcal{N}$ is the data loading cost, implementations without a KB suffer, resulting in $\mathcal{O(C * N)}$ complexity. With a KB, data loading is necessary only once, enhancing the complexity to $\mathcal{O(C + N)}$.

 In this paper, we propose the addition of a knowledge base (KB) to providing an optimized representation of the trace logs over which the declarative models $\mathcal{M}$ are going to be queried. % By storing it in a KB,  data is obtained and processed only once, rather than per query, as existing state of the art. 

\end{document}

