\documentclass[11pt]{article}
\setlength{\rightmargin}{2.5cm}
\setlength{\leftmargin}{2.5cm}
\usepackage{setspace}
\onehalfspacing
\textwidth 6.5in
\setlength{\topmargin}{0in}
\setlength{\headheight}{0in}
\setlength{\headsep}{0in}
\setlength{\textheight}{9in}
\setlength{\oddsidemargin}{0in}
\setlength{\textwidth}{6.2in}
\usepackage{xcolor}

\usepackage[english]{babel}

%%%% COMMENTS
%% Comments package -- Giacomo
\usepackage{pdfcomment}
\usepackage{environ}
%{\pdfcomment{\BODY}}
\usepackage{easyReview}
\makeatother

\usepackage{newclude}


%% Swiggly notes -- Giacomo
\usepackage{xparse}
\usepackage{marginnote}
\usepackage{soul}
\usepackage{lipsum}
\usepackage{tikz}
\usetikzlibrary{calc}
\usetikzlibrary{decorations.pathmorphing}
\newlength\LineWidth
\newlength\Amplitude
\newlength\SegLength
\setlength\LineWidth{0.4pt}
\setlength\Amplitude{1pt}
\setlength\SegLength{5pt}
\definecolor{HLcolor}{RGB}{240,0,0}
\newcommand\tikzmark[1]{%
	\tikz[overlay,remember picture] \node (#1) {};}
\makeatletter
\newcommand{\highlight@DoHighlight}{
	\draw[HLcolor,line width=\LineWidth,decorate,decoration={zigzag,amplitude=\Amplitude,segment length=\SegLength}]  ($(begin highlight)+(0,-2pt)$) -- ($(end highlight)+(0,-2pt)$) ;
}
\newcommand{\highlight@BeginHighlight}{
	\coordinate (begin highlight) at (0,0) ;
}
\newcommand{\highlight@EndHighlight}{
	\coordinate (end highlight) at (0,0) ;
}
\newdimen\highlight@previous
\newdimen\highlight@current
\DeclareRobustCommand*\highlight[1][]{%
	\SOUL@setup
	%
	\def\SOUL@preamble{%
		\begin{tikzpicture}[overlay, remember picture]
			\highlight@BeginHighlight
			\highlight@EndHighlight
		\end{tikzpicture}%
	}%
	%
	\def\SOUL@postamble{%
		\begin{tikzpicture}[overlay, remember picture]
			\highlight@EndHighlight
			\highlight@DoHighlight
		\end{tikzpicture}%
	}%
	%
	\def\SOUL@everyhyphen{%
		\discretionary{%
			\SOUL@setkern\SOUL@hyphkern
			\SOUL@sethyphenchar
			\tikz[overlay, remember picture] \highlight@EndHighlight ;%
		}{%
		}{%
			\SOUL@setkern\SOUL@charkern
		}%
	}%
	%
	\def\SOUL@everyexhyphen##1{%
		\SOUL@setkern\SOUL@hyphkern
		\hbox{##1}%
		\discretionary{%
			\tikz[overlay, remember picture] \highlight@EndHighlight ;%
		}{%
		}{%
			\SOUL@setkern\SOUL@charkern
		}%
	}%
	%
	\def\SOUL@everysyllable{%
		\begin{tikzpicture}[overlay, remember picture]
			\path let \p0 = (begin highlight), \p1 = (0,0) in \pgfextra
			\global\highlight@previous=\y0
			\global\highlight@current =\y1
			\endpgfextra (0,0) ;
			\ifdim\highlight@current < \highlight@previous
			\highlight@DoHighlight
			\highlight@BeginHighlight
			\fi
		\end{tikzpicture}%
		\the\SOUL@syllable
		\tikz[overlay, remember picture] \highlight@EndHighlight ;%
	}%
	\SOUL@
}
\makeatother
\DeclareDocumentCommand\MarkText{O{red}O{1pt}O{5pt}m}{%
	\colorlet{HLcolor}{#1}
	\setlength\Amplitude{#2}%
	\setlength\SegLength{#3}%
	\tikzmark{endquote}\tikzmark{beginquote}\highlight{#4}%
}

%%% Pseudocodes
\usepackage{inconsolata}
\usepackage[noend]{algpseudocode}
%% Multiline
\newcommand\CONDITION[2]%
{\begin{tabular}[t]{@{}l@{}l@{}}
		#1&#2
	\end{tabular}%
}
\algdef{SE}[WHILE]{While}{EndWhile}[1]%
{\algorithmicwhile\ \CONDITION{#1}{\ \algorithmicdo}}%
{\algorithmicend\ \algorithmicwhile}
\algdef{SE}[FOR]{For}{EndFor}[1]%
{\algorithmicfor\ \CONDITION{#1}{\ \algorithmicdo}}%
{\algorithmicend\ \algorithmicfor}
\algdef{S}[FOR]{ForAll}[1]%
{\algorithmicforall\ \CONDITION{#1}{\ \algorithmicdo}}
\algdef{SE}[REPEAT]{Repeat}{Until}{\algorithmicrepeat}[1]%
{\algorithmicuntil\ \CONDITION{#1}{}}
\algdef{SE}[IF]{If}{EndIf}[1]%
{\algorithmicif\ \CONDITION{#1}{\ \algorithmicthen}}%
{\algorithmicend\ \algorithmicif}%
\algdef{C}[IF]{IF}{ElsIf}[1]%
{\algorithmicelse\ \algorithmicif\ \CONDITION{#1}{\ \algorithmicthen}}
%% End Multiline
\usepackage{algorithm,algorithmicx}
\makeatletter
\algrenewcommand\ALG@beginalgorithmic{\ttfamily}
\makeatother
\usepackage{adjustbox} %% Fitting to page
\usepackage{amsmath}
%% mathematics
\usepackage{braket} %% Set, tuple...



\usepackage{amsmath,amssymb}
\newcommand{\Next}{\ensuremath{\mathbf{X}}}
\newcommand{\Globally}{\ensuremath{\mathbf{G}}}
\newcommand{\Future}{\ensuremath{\mathbf{F}}}
\newcommand{\WeakUntil}[2]{\ensuremath{{#1}\;\mathbf{W}\;{#2}}}
\newcommand{\DUntil}[2]{\ensuremath{{#1}\;\mathbf{U}\;{#2}}}
\newcommand{\MonoDeclareClause}[4]{\textsf{#1}(\texttt{#2},#3,{#4})}
\newcommand{\DeclareClause}[5]{\textsf{#1}(\texttt{#2},#3,\texttt{#4},#5)}
\newcommand{\DeclareClauseWithJoin}[6]{\textsf{#1}(\texttt{#2},#3,\texttt{#4},#5)\;\textsf{where}\;#6}
\newcommand{\Sdeclare}[3]{\DeclareClause{#1}{#2}{\textbf{true}}{#3}{\textbf{true}}}
\newcommand{\LTLf}{\textup{LTL}\textsubscript{f}\;}


\begin{document}
%\setlength{\rightmargin}{2.5cm}
%\setlength{\leftmargin}{2.5cm}
%\setleftmargin{2.5cm}	
%\setrightmargin{2.5cm}	
\title{Running Temporal Logical Queries on Knowledge Bases}

\author{[Authors Removed For Double Blind Review]}

\maketitle
~\\
\begin{abstract}
State of the art approaches on model checking for declarative temporal models exploit straightforward and computationally intensive solutions, where data needs to be scanned multiple times per temporal model (query). By exploiting adequate knowledge base representation as well as ad hoc query plans, data loading and indexing is necessary only once, thus enhancing the complexity and allowing queries to be run multiple times after loading the traces only once. In fact, no solution exploiting efficient relational databases, querying multiple declarative constraints, and expressing temporal correlation conditions as relational join operations is known. The resulting system can be then exploited in real-time scenarios, such as video games and cyber-security systems. For efficiently exploiting such a KB, this paper proposes for the first time, to the best of our knowledge, a specialised set of operators determining both the semantics of the temporal model and the operators for the query plan. As our trace logs are data-aware, we make also such operators support join conditions for the first time. Experiment shows\dots
\end{abstract}
~\\

{\bf Keywords}: Artificial Intelligence, Knowledge Bases, Query Plan, Finite Temporal Logic
~\\

% A category with the (minimum) three required fields
{\bf Topics} Physical Layout, Algorithms
~\\


\include*{sections/10introduction}
\include*{sections/20related}
\include*{sections/21temporallogicoperators}
\include*{sections/22conformance}
\include*{sections/30declare_definition}
\include*{sections/40physical_model}
\include*{sections/45operator_semantics}
\include*{sections/50experiments}
\include*{sections/60conclusions}




\bibliographystyle{abbrv}  
\bibliography{refs.bib}  


\end{document}

