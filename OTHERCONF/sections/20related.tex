\section{Related Work}
%\textit{In this space, you usually want to quote the papers that you know that are near to the area. You want to make comparison, make similarities, and state which are their deficiencies if any} Questions that a reviewer might ask:
%\begin{itemize}
%	\item Is the current literature being assessed exhaustive, or is there something missing?
%	\item If the current literature us exhaustive, is it clear why each component is introduced?
%	\item For each piece of literature, are the different approaches compared and analysed, in weak and strong links? Which are the connections to your proposed approach?
%	\item E.g., are the opponents introduced as baselines for the experiment section also introduced in here?
%\end{itemize}


\paragraph*{XES Log Model}\label{sec:XES}

%\texttt{\color{red}[TODO: simplify, remove unnecessary logical notation] } 
(Data) \textit{payloads} are maps  associating attributes (i.e., \textit{keys}) to data values. %We consider also the case in which the value of a certain key $k$ is missing in a payload. In particular, we denote as $\varepsilon$ an element $\varepsilon\notin V$, such that $p(k)=\varepsilon$ for $k\notin\textup{dom}(p)$. 
Given a finite set of activity labels $\textsf{Act}$, an event $\sigma_j^{i}$ is a pair $\Braket{\textsf{a},p}$, where $\textsf{a}\in\textsf{Act}$ is an activity label, and $p$ is a payload. %; we denote with $\lambda$ (and $\varsigma$) the first (and second) projection of such pair, i.e., $\lambda(\sigma_j^{i})=\texttt{A}$ (and $\varsigma(\sigma_j^{i})=p$). 
A \textit{trace} $\sigma^i$ is a temporally-ordered and finite sequence of distinct events $\sigma^i=\sigma_1^i\cdots\sigma_n^i$, modeling a process run. %We distinguish the trace keys ($K_t$) from the event keys ($K_e$), such that $K=K_t\cup K_e$ with $K_t\cap K_e=\emptyset$: 
all events within the same trace associate the same values to the same trace keys. %, i.e., $\forall \Braket{\texttt{A}_i,p_i},\Braket{\texttt{A}_j,p_j}\in\sigma.\;\forall k\in K_t.\; p_i(k)=p_j(k)$. 
A log $\mathcal{L}$ is a finite set of traces $\Set{\sigma^1,\dots,\sigma^m}$. We denote  $\Sigma\subseteq\textsf{Act}$ as the set of all the possible activity labels in the log. If traces also contain a payload, then this can be easily mimicked by adding an extra event containing such a payload, \textsf{\_\_trace\_payload}, at the beginning of the trace. This  characterization \cite{bpm21} is compliant with the \textsc{eXtensible Event Stream} (XES) format, which is the \textit{de facto} standard for representing event logs within the Business Process Management community \cite{XES}. Please refer to \S\ref{ssec:dl} for the representation of each log as a separated in-memory database.