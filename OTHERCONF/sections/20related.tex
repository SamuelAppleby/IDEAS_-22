\section{Related Work}
%\textit{In this space, you usually want to quote the papers that you know that are near to the area. You want to make comparison, make similarities, and state which are their deficiencies if any} Questions that a reviewer might ask:
%\begin{itemize}
%	\item Is the current literature being assessed exhaustive, or is there something missing?
%	\item If the current literature us exhaustive, is it clear why each component is introduced?
%	\item For each piece of literature, are the different approaches compared and analysed, in weak and strong links? Which are the connections to your proposed approach?
%	\item E.g., are the opponents introduced as baselines for the experiment section also introduced in here?
%\end{itemize}


\paragraph*{XES Log Model}\label{sec:XES}


(Data) \textit{payloads} are maps  associating attributes (i.e., \textit{keys}) to data values. 
Given a finite set of activity labels $\textsf{Act}$, an event $\sigma_j^{i}$ is a pair $\Braket{\textsf{a},p}$, where $\textsf{a}\in\textsf{Act}$ is an activity label, and $p$ is a payload, mapping each key to a single value. 
A \textit{trace} $\sigma^i$ is a temporally-ordered and finite sequence of distinct events $\sigma^i=\sigma_1^i\cdots\sigma_n^i$, modelling a process run. 
All events within the same trace associate the same values to the same trace keys. 
A log $\mathcal{L}$ is a finite set of traces $\Set{\sigma^1,\dots,\sigma^m}$. We denote  $\Sigma\subseteq\textsf{Act}$ as the set of all the possible activity labels in the log. If traces also contain a payload, then this can be easily mimicked by adding an extra event containing such a payload, \textsf{\_\_trace\_payload}, at the beginning of the trace. This  characterization \cite{bpm21} is compliant with the \textsc{eXtensible Event Stream} (XES) format, which is the \textit{de facto} standard for representing event logs within the Business Process Management community \cite{XES}. 