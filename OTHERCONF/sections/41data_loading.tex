\subsection{Data Loading}\label{ssec:dl}
The data loading phase allows to load logs that are serialized in multiple possible formats, thus uncluding the XML-based XES standard, a tab-separated file representation - where each line represents a trace and each column represents a single activity label -, and the \textsc{Human Readable Log Format} firstly introduced in \cite{bpm21}. For these, we use different data parsers, which are still linked to the same data loading primitives. 

If the log does not come with data payloads, the entire log can be represented into two relational tables, \textsf{CountingTable(ActId,TraceId,Count)} and \textsf{ActivityTable(ActId,TraceId,EventId,\\PrevEvent,NextEvent)}. While the former counts the occurrence of each activation label in $\Sigma$ in each trace, the latter lists all of the possible events similarly to SQLMiner. Both tables compactly represent the initial three columns as a 64-bit unsigned integer, which is also used to sort the tables in ascending order. Each row $\braket{\textsf{a},j,k}$ in the \textsf{CountingTable} represents that the activity $\textsf{a}$ appears $k$ times in the trace $\sigma^j$, while each row $\braket{\textsf{a},j,k,p,q}$ in the \textsf{ActivityTable} represents that the $k$-th event of the $j$-th trace ($\sigma^j_k\in \sigma^j$) is labelled as $\textsf{a}$ ($\lambda(\sigma^j_k)=\textsf{a}$), and $p$ is the pointer to the immediately preceding ($\sigma^j_{k-1}$) and following ($\sigma^j_{k+1}$) events within the trace, if any.  \texttt{\color{red}[TODO: link to the infographic picture, continue the use case example]}. 

If, on the other hand, the log is associated to either trace or event payloads, we exploit the very well-known column-based storage \cite{IdreosGNMMK12}, thus representing the values $v$ associated to a same payload key $k$ as rows belonging to one single table,  \textsf{AttributeTable$k$}. Furthermore, each value $v$ is linked to the event or trace containing it via a foreign key or, in our case, via a table offset. In our implementation, each row $\braket{\textsf{a},v,i}$ of the  \textsf{AttributeTable$k$(ActId,Value,Offset)} table represents a value $v$ associated to the key $k$ from the event $\sigma^j_k$ appearing in the $i$-th row of the \texttt{ActivityTable}. In order to perform payload-based queries efficiently, the table is sorted in ascending order by the whole three columns, so that data range queries can be effectively run by exploiting binary search algorithms. \texttt{\color{red}[TODO: link to the infographic picture, continue the use case example]}. 

\textsf{CountingTable} is mainly accessed for existential and \textsf{Exists} and \textsf{Absence} templates where no data payload is specified, while the \textsf{ActivityTable} is mainly used for returning all of the events within the log associated to a given activity label, or returning all of the events happening at either the beginning or at the end of a trace. Each table \textsf{AttributeTable$k$}, on the other hand, is accessed for returning all the events satisfying a given condition associated to a specific data key. As each data condition is always associated to a given activity label, each value is grouped by activity label via table ordering.

After loading the whole dataset, both the number of the traces within the log ($|\mathcal{L}|$) and the number of distinct activity labels ($|\Sigma|$) is known. Therefore, we can determine the number of occurrences of each $i$-th activity label in $\Sigma$ for each trace $\sigma^j\in\mathcal{L}$ by accessing the rows from $(|\mathcal{L}|+1)\cdot i$ to $(|\mathcal{L}|+1)\cdot i+
|\mathcal{L}|-1$ without the need of any external indexing data structure. \texttt{\color{red}[TODO: link to the infographic picture, continue the use case example]}. On the other hand, the loading and and indexing phase generates an \textsf{ActivityTable} associated to two indices, a primary index and a secondary index. While the former allows to effectively return all of the events associated to a specific activity label, the latter is used to access to either the first and to the last event in a trace.  \texttt{\color{red}[TODO: link to the infographic picture, continue the use case example]}

Details of the loading and indexing phase are omitted due to the page limits.



