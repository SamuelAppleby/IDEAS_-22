\section{Logical Model}

\subsection{(Intermediate) Result Representation}
Within the computation pipeline, (intermediate) results are represented as a set of triplets $\braket{i,j,L}$ representing that, starting from event $\sigma^i_j$ in trace $\sigma^i$, we might observe activation, target, or correlation conditions in $L$, an ordered vector.  While for activation and target 
we 
preserve 
the matched event id,
correlations keep track of
both the activation and the target condition leading to the satisfaction of a given $\Theta$ predicate (see the next section). This is a sensible representation, as per declarative constraints, it may exist only one possible $\Theta$ predicate. Such triplets are sorted by trace id and event id, and operators manipulating those (\S\ref{sec:xltlf}) guarantee that only one triplet should appear per unique trace and event id. This guarantees efficient join operations across different intermediate results, as well as efficient counting of the satisfied conditions for each trace. %While the former might be implemented via sorted joins, the latter can be achieved by either binary search operations or by look-ahead jumps, as the events pertaining to a trace are contiguous. 
E.g., Clause \Circled{\small C} from \figurename~\ref{fig:knobab_pipeline} requires access to just \textsf{AttributeTable}s, as all of the activity labels are associated to data conditions. When we want to return events for which $\mathcal{P}_{12}$ holds, we need to only consider the data associated to \texttt{Lumpectomy} events having a positive biopsy and levels of CA15.3 greater than 50. This will require the intersection of the events related to biopsy with the ones related to CA15.3. The selected rows are then  converted into the intermediate result representation ad intersected; in this situation, we only obtain $\Set{\braket{3,2,\{A(2)\}}}$, as the only event meeting such requirements is the second from the third trace. 
As we are going to see in the next paragraph, A is the container of matched activation conditions. 
Similarly, $\mathcal{P}_{8}$ will return $\Set{\braket{2,2,\{A(2)\}}}$, thus obtaining $\Set{\braket{1,2,\{A(2)\},\braket{3,2,\{A(2)\}}}}$ as a final result associated to \Circled{\small C}: this remarks that only traces \ding{192} and \ding{194} describe patients that underwent a surgical operation under such conditions.

Our proposed representation is different from the one provided by \cite{BurattinMS16} which cannot represent for each event within a trace all the possible activation, target, or join condition happening in the future, as it is impossible to represent single trace events that are not necessarily represented by activation or target conditions. As observed in \S\ref{sec:DAD}, this information is required for checking the satisfiability of $\varphi$ while jointly visiting both the trace (now represented as subsequent rows in the result representation) and the formula. 
In fact, authors exploit
a hash map of hash maps, associating  each trace to the collected activation conditions which, in turn, might be associated with further target conditions. 
This solution is even less efficient than exploiting sorted linear data structures.



\subsection{eXTended \LTLf operators}\label{sec:xltlf}

We now propose the extended \LTLf operators (\xLTLf) directly exploited by our pipeline:
\[\begin{split}
\phi:=&\;\;\;\textsf{Init}_{A/T}(A,p)\gsep\textsf{End}_{A/T}(A,p)\gsep\textsf{Exists}_{A/T}(n,A,p)\gsep\textsf{Absence}_{A/T}(n,A,p)\\
     &\gsep \textsf{Next}(\phi)\gsep \textsf{Globally}(\phi)\gsep \textsf{Future}(\phi)\gsep \textsf{Not}(\phi)\\
     &\gsep \textsf{Or}(\phi,\phi',\Theta)\gsep \textsf{And}(\phi,\phi',\Theta)\gsep \textsf{Until}(\phi,\phi',\Theta)\\
     &\gsep \textsf{AndGlobally}(\phi,\phi',\Theta)\gsep \textsf{AndFuture}(\phi,\phi',\Theta)\gsep \textsf{AndNextGlobally}(\phi,\phi',\Theta)\\
\end{split}\]


\begin{algorithm}
	\caption{\xLTLf pseudocode implementation for the basic timed operators}\label{algo:xltlfAlgo}
	
	
	\begingroup % trick algorithm2e into thinking we're in one column mode
	\csname @twocolumnfalse\endcsname
	\noindent
	\resizebox{\textwidth}{!}{%
		\begin{minipage}{1.4\textwidth}
			%%%%%%%%%%%%%%
			\begin{algorithmic}[1]
				\Statex
				\Function{Future}{$\phi$}
				%\State {\textbf{Require:\;}sorted($\phi$)}
				\ForAll{$\braket{t,e,L}\in\phi$}
				\textbf{yield} $\braket{t,e,\bigcup\Set{L'|\braket{t,e',L'}\in\phi\texttt{\textbf{\;and\;}}e'\geq e}}$
				\EndFor
				\EndFunction\medskip
				%%
				\Function{Globally}{$\phi$}
				%\State {\textbf{Require:\;}sorted($\phi$)}
				\ForAll{$\braket{t,e,L}\in\phi$}
				\State $E\gets\Set{e'|\braket{t,e',L'}\in\phi\texttt{\textbf{\;and\;}}e'\geq e}$
				\If {$|E|=\ell_t-e$} 
				\textbf{yield} $\braket{t,e,\bigcup\Set{L'|\braket{t,e',L'}\in\phi\texttt{\textbf{\;and\;}}e'\in E}}$
				\EndIf
				\EndFor
				\EndFunction\medskip
				%%
				\Function{Next}{$\phi$}
				%\State {\textbf{Require:\;}sorted($\phi$)}
				\ForAll{$\braket{t,e,L}\in\phi$\textbf{\;s.t.} $e>1$}
				\textbf{yield} $\braket{t,e-1,L}$
				\EndFor
				\EndFunction\medskip
				%%
				\Function{CommonJoin}{$\phi,\phi',\Theta,{isDisjunctive}$} %\Comment{Sort Merge (Full-Outer-)Join}
				%\State %{\textbf{Require:\;}sorted($\phi$)\textbf{\;and\;}sorted($\phi'$)}
				\State  $it\gets$\textbf{Iterator}$(\phi), it\gets$\textbf{Iterator}$(\phi')$
				\While{$it\neq\emptyset$\textbf{\;and\;}$it'\neq\emptyset$}
				\State $\braket{t,e,L}\gets\texttt{current}(it)$, $\braket{t',e',L'}\gets\texttt{current}(it')$
				
				
				
				\If{$t=t'$ \textbf{and} $e=e'$}
				\State $L''\gets\emptyset$
				
				\If{$\Theta\neq\textbf{true}$\;\textbf{and}\;$L\neq\emptyset$\;\textbf{and}\;$L'\neq\emptyset$}
				\ForAll{$A(m)\in L$\textbf{\;and\;}$T(n)\in L'$\textbf{\;s.t.\;}$\Theta(m,n)$}
				\State $L''\gets L''\cup\Set{M(m,n)}$
				\EndFor
				\Else
				\State \textbf{if}\;$L=\emptyset$\;\textbf{then}\;$L''\gets\{A(e)\}$\;\textbf{else}\;$L''\gets L$ \State \textbf{if}\;$L'=\emptyset$\;\textbf{then}\;$L''\gets L\cup\{T(e')\}$\;\textbf{else}\;$L''\gets L''\cup L'$
				\EndIf
				
				\State \textbf{if}\;$L''\neq\emptyset$\;\textbf{then\;yield} $\braket{t,e,L''}$; 
				\State $\texttt{next}(it)$; $\texttt{next}(it')$; 
				\ElsIf{$t<t'$\textbf{\;or\;}($t=t'$\textbf{\;and\;}$e<e'$)} 
				\If  {$isDisjunctive$} \textbf{yield} $\braket{t,e,L}$  \EndIf
				\State $\texttt{next}(it)$
				\Else 
				\If  {$isDisjunctive$} \textbf{yield} $\braket{t',e',L'}$  \EndIf
				\State $\texttt{next}(it')$
				\EndIf
				\EndWhile
				\EndFunction\medskip
				%%
				\Function{And}{$\phi,\phi',\Theta$}
				\Call{CommonJoin}{$\phi,\phi',\Theta,\textbf{false}$}
				\EndFunction\medskip
				%%
				\Function{Or}{$\phi,\phi',\Theta$}
				\Call{CommonJoin}{$\phi,\phi',\Theta,\textbf{true}$}
				\EndFunction\medskip
				%%
				\Function{Until}{$\phi,\phi',\Theta$}
				%\State %{\textbf{Require:\;}sorted($\phi$)\textbf{\;and\;}sorted($\phi'$)}
				\ForAll{$t$ \textbf{\;s.t.\;} $\braket{t,i',L'}\in \phi'$}
				\State $\alpha\gets 1$; $Map\gets\{\}$; $i\gets \min_{\iota}\braket{t,\iota,L}\in \phi'$; $I\gets \max_{\iota}\braket{t,\iota,L_I}+1$
				\While{$i< I$}
				\If{$\alpha=i$} \State $\textit{Map}[\alpha]\gets\textit{Map}[\alpha]\cup L'$
				
				\State $i\gets \min_{\iota,\iota>i}\braket{t,\iota,L}$
				\ElsIf{\textbf{exists} $\braket{t,j,L_j}\in \phi$ \textbf{s.t.}\; $j<i$}
				\If{$\braket{t,\alpha,L_\alpha},\braket{t,\alpha+1,L_{\alpha+1}},\dots,\braket{t,i-1,L_{i-1}}\in\phi,$\\\textbf{\;and\;}$\Theta(i,j)$\textbf{\;for all\;}$T(j)\in L_\alpha\cup\dots\cup L_{i-1}$}
				\State $Map[\alpha]\gets Map[\alpha]\cup \Set{M(k,i)|T(k)\in L_\alpha\cup\dots\cup L_{i-1}}$
				\State $i\gets \min_{\iota,\iota>i}\braket{t,\iota,L}\in\phi'$
				\Else \;$\alpha\gets \alpha+1$
				\EndIf
				\Else \;$\alpha\gets i$
				\EndIf
				\EndWhile
				\ForAll{$\braket{i,L}\in \textit{Map}$}
				\textbf{yield} $\braket{t,i,L}$
				\EndFor
				\EndFor
				\EndFunction\medskip
			\end{algorithmic}
			%%%%%%%%
		\end{minipage}%
	}% <------------- end of \resizebox
	\endgroup
\end{algorithm}
Operators in the first line filter traces' events and represent these into the previously-described result representation. 
\textsf{Init} (\textsf{End}) returns the events at the beginning (end) of each trace satisfying the condition $A\wedge p$. Similarly to \cite{BurattinMS16}, each of these operators might be expressed as either an untimed or as a timed specification. Any operator will be considered timed by default when appearing inside a timed operator, like \textsf{Next}, \textsf{Globally}, \textsf{Future}, \textsf{Until}, and any other composed operator from the last line. E.g., In \figurename~\ref{fig:knobab_pipeline}, \textsf{Exists}$(1,\mathcal{P}_3)$ is a shorthand for \textsf{Exists}$(1,\textrm{FollowUp},-\infty<\textup{CA-15.3}<23.5)$, as each atom always associates an activity label to a payload condition. The operator associated to \textsf{Absence}$(1,\mathcal{P}_3)$ is untimed, while the \textsf{Exists}$(1,\mathcal{P}_3)$ descendant of \textsf{Globally} is timed.  While the timed definition returns a tuple $\braket{i,j,L}$ for each possible event $\sigma^i_j$ within the trace $\sigma^i$ where the formula holds, the untimed specification only checks whether the formula holds at the beginning of the trace. E.g., untimed \textsf{Exists} (\textsf{Absence}) returns the first event trace if at least $n$ (at most $n-1$) events satisfy $A\wedge p$, while the timed version returns the events satisfying (not satisfying) $A\wedge p$ (always $n=1$). All of these operators might be optionally marked as returning either an activation ($A$) or a target ($T$) condition, so that each  $\braket{i,j,L}$ triplet has $L=\{A(j)\}$ or $L=\{T(j)\}$; when no mark is specified, $L$ is empty. To wrap up the previous example, the timed  \textsf{Exists}$(1,\mathcal{P}_3)$ will list the events where $\mathcal{P}_3$ happened, $\Set{\braket{1,3,\{A(3)\}}}$, while the untimed version will just list the traces where such event happened and collect the event of interests in $L$, $\Set{\braket{1,1,\{3\}}}$.


The next two lines report the same operators described in \S\ref{sec:DAD} with the addition of the explicit correlation conditions over activation and target conditions for each binary operator. Algorithm~\ref{algo:xltlfAlgo} provides  implementations of the timed versions of such operators, due to lack of space untimed versions are not provided, yet available in our codebase: please observe that $\textsf{Next}(\phi)$ keeps unaltered the activation and target conditions from $\phi$ and just returns the events where $\phi$ happens as a subsequent step. Any binary operator supports $\Theta$ conditions:  \textsf{And} (and \textsf{Or}) can be expressed as a (full-outer-)$\Theta$-join algorithm over the activation and target conditions stored in $L$ associated with the same event. If at least one activation condition matches one target condition from the same event, those are expressed as a marked correlation condition $M(i,j)$ which is then returned by the join. Regarding the same \textsf{Choice} clause from \figurename~\ref{fig:knobab_pipeline}, the correlation condition $\Theta$ associated to \textsf{Or} is then computed for each activation/target match, and if the condition is passed, the resulting match is added to $L$. 


The remaining operators merge multiple operators together when 
a specific implementation outperforms
the execution of the operators separately:
 e.g., $\textsf{AndFuture}(\phi,\phi',\Theta)$ is equivalent to\\ $\textsf{And}(\phi,\textsf{Future}(\phi'),\Theta)$, but preliminary experiments reveal that the former has a more efficient implementation than computing the latter. This choice was inspired by relational algebra, where $\theta$-joins are usually more efficient than performing a join and a selection operation separately.
 On the other hand, % Similarly,  we can associate the correlation condition to implication operators after rewriting 
 $\textsf{Implies}(\phi,\phi',\Theta)$ is rewritten as $\textsf{Or}(\textsf{Not}(\phi),\,\textsf
{And}(\phi,\phi',\Theta),\,\textbf{true})$. As per previous discussion, the left leaf of \textsf{AndFuture}$_\Theta$ in \figurename~\ref{fig:knobab_pipeline} returns all of the referral events with CA 15-3 above the safeguard levels, $\Set{\braket{1,1,\{A(1)\}},\braket{3,1,\{A(1)\}}}$, while the right leaf returns just the follow-up events below such levels, $\Set{\braket{1,3,\{A(3)\}}}$. The operator \textsf{AndFuture}$_\Theta$ will then return only $\Set{\braket{3,1,\{M(1,3)\}}}$, as only the first trace will have a decrease below the safeguard levels from referral to follow-up.
%
Each \xLTLf operator is going to both return and/or accept data in the result representation, thus making such operators closed on such format.





