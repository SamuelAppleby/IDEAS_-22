\vspace{-4mm}\section{Conclusions and Future Works}
We propose KnoBAB, a fully relational database architecture for computing Conformance Checking via conjunctive queries, as well Max-SAT and clause \textsc{Confidence}/\textsc{Support} functions.  KnoBAB consists of a data loader and indexer, query compiler, and an execution engine, thus fully matching the architecture of a relational database. This solution was enabled by the extension of the traditional \LTLf operators, providing algebraic semantics to declarative temporal models, so as to support data operations over tuples representing trace  events. Our solution is not limited to one single declarative language of choice, as it might support any possible model that can be expressed via \xLTLf operators. Based on the latest solutions in current database literature, the query plan was also designed to minimize the data access by running the common sub-queries at most once.
%
KnoBAB outperforms state of the art solutions both tailored to the specific dataset or based on traditional relational databases running SQL queries.  This solution will enable us to learn models exploiting abductive reasoning rather than traditional mining techniques, thus also providing safety guarantees over noisy data and models that are inconsistency free \cite{PicadoDTL20}. 

Future works will provide extensive benchmarks for bigger log datasets and will provide speed-up results for the parallelized execution of the resulting query plan: despite this being already implemented, we postpone those results due to the lack of space in the present paper. For the time being, the logs available from the research community are quite compact, and therefore the whole dataset is well fit in primary memory. Dealing with actual big data solutions or bigger models will require us to migrate the data store location to secondary memory, thus requiring the adoption of Near-Data Processing techniques \cite{GuYBJLYKKYCJC16}. 

%Last, 
The adoption of relational databases and operator-based query plans might enable incremental trace updates so to extend those at runtime: this open research problem  can be now  solved by exploiting algebraic rewriting rules similar to the ones from relational databases, thus requiring a formal definition of \xLTLf operators. 

%\textit{Summarize the abstract even more, as now you need also to summarize the outcome of the experiments. Furthermore, say what was legitimately left out, and which are the future works being scheduled for extending the present work.}
%
%In particular:
%\begin{itemize}
%	\item Are all of the open questions in the introduction closed at this point? Are all the questions answered? 
%	\item If something relevant is left out, is that considered for future work?
%\end{itemize}
%
%Experiments on parallelizations are omitted due to lack of space. Are implemented, but they are going to be tested in our future work.
%
%Talk about the support of runtime traces (require much less computation than data mining - many fewer clauses, much less data (only 1 trace))
%
%Another possible addition could be the inclusion of external data conditions that are not bound to data payloads. We currently consider target, correlation and theta conditions, but not external factors. 
%
%Therefore, we can represent: 
%
%\emph{`If it has started raining, and the moisture content of the soil is above 50 percent, a flood check must happen if more rain is forecast'}.  
%As a Declare clause, this would formulate:
%
%{\tiny$\mathbf{Response(RainStart \{moistureContent>50\%\}, FloodCheck\{forecast=rain\}) WHERE RainStart\{location\} = FloodCheck\{location\}}$}
%
%What KnoBAB does not currently support, however, is the addition of an external global condition. 
%
%Therefore a constraint such as:
%
%\emph{`If it has started raining, and the moisture content of the soil is above 50 percent, a flood check must happen \textbf{2 days after} if more rain is forecast'}.  
%
%Is not currently supported. The inclusion of these conditions would allow for greater expressiveness of each clause, and therefore the declarative model itself.