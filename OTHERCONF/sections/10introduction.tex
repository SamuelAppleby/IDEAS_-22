\section{Introduction}

%\paragraph*{Brief introduction of your general area of interest: provide the \textbf{context} to the overall general setting.} 
\textit{Conformance checking} is an integral part of \textsc{Artificial Intelligence} \RevRepl{that bridges}{bridging} data mining and business process management \cite{bpm21}. It assesses whether a sequence of distinguishable events (i.e., a \textit{trace}) conforms to the expected process behaviour represented as a \textit{process model} \cite{RozinatA08}. \RevRepl{Due to this temporal requirement, modelling languages that support this information, such as Declare, must be used to express such clauses.}{These constraints can be expressed by either multiple declarative human readable templates that should be satisfied simultaneously (i.e., \textit{conjunctive query})} \cite{Li2020}\RevAdd{, where each one of these represent a different temporal behavioural pattern, or equivalently as graph models} \cite{MultiPerspective}. %Such a model might be either represented as a set of temporal clauses, determining correlations between events happening at a previous time of the trace (\textit{activation}) and others happening in the immediate future (\textit{target}).
\RevAdd{Despite the two approaches are equivalent}  \cite{AgostinelliBFMM21}, \RevAdd{in this paper we  focus on the former representation, as the latter is inefficient for conformance checking of data-aware models} \cite{bpm21}. Such declarative temporal rules are not limited to the mere presence of specific events within the trace, but also determine \RevRepl{how such clauses might occur}{temporal occurrence patterns}.  \RevRepl{Declarative models can be generated from these clauses, which can}{Such models are usually exploited as the core} behind an AI's temporal decision making: \RevDel{We can vary the data by which the clauses are generated to give unique behaviour.}\PopUpComment{Giacomo}{Unclear}mining a hospital log consisting of successful traces (correct procedures, adequate medication etc.) could produce a model representing the patterns leading to `good' outcomes  \cite{Amantea2020}. Such mining approach could retrieve  declarative models of interest \cite{mining}. An AI could then query \RevRepl{this}{the resulting mined} model \RevRepl{for a current trace to provide the action}{for determining} which \RevAdd{clinical situation} is likely to \RevRepl{be  the next correct decision}{adhere to the expected clinical standards}.



When a trace does not adhere to the model, we say that the trace is \textit{deviant} \cite{bpm21}. \RevDel{In the case of traces, the order of the events is important, and temporal information of each event must be stored.}\PopUpComment{Giacomo}{This is good, I also added the reference claiming that declarative models can be also equivalently expressed as graphs.} E.g., Given a Declare template \textsf{RespondedExistence}, \DeclareClause{RespondedExistence}{A}{\textbf{true}}{B}{\textbf{true}} is the instantiated Declare clause for event labels \texttt{A} and \texttt{B} that states \emph{`If event A happens, event B must happen sometime in the future'}: where A (B) is the \textit{activation} (\textit{target}) condition; a trace would be a deviant \emph{iff.} a trace contained an instance of A that was not followed by an instance of B. \RevAdd{Furthermore, \textbf{true} predicates associated to activation (target) conditions can be enriched to express data conditions (\S\ref{sec:DAD}). As any declarative language, the specification of such templates can be expressed as a set of logical operators: in this field, Finite Liniear Time Logic (\LTLf) is usually exploited:} for insatnce $\DeclareClause{RespondedExistence}{A}{p}{B}{q}$ becomes $\Future(A\wedge p) \Rightarrow \Future(B\wedge q)$, \RevAdd{where $\Future(A\wedge p)$ express the intention that a trace labelled as $A$ satisfying a data condition $p$ should occur any time within the trace }\cite{bpm21}\RevAdd{; such activation and target conditions might be also extended with correlation conditions, thus expressing $\Theta$-joins between the payload associated to the activation event to the one associated to the target event.}
%%{\color{red}[TODO: replace] To further decompose these clauses, algebraic notation can be used to represent the set of operators that are the constituents of a given clause. Declare templates can be represented using \LTLf \cite{Li2020}. This allows for a flexible conformance checking implementation, as each clause can be represented as a unique pairing of \LTLf operators and join operators, for insatnce $\mathbf{RespondedExistence(A,B)}$ becomes $\Future(A) \Rightarrow \Future(B)$\PopUpComment{Giacomo}{Please observe that mathcal should be used only by surrounding the symbol of interest. Otherwise, in some other scenarios, you might have faults. Still, we are going to use a box/diamond notation for this paper. I added some macros for that}. As part of the process mining pipeline, conformance checking is used to identify patterns emerging from a given log. Therefore, process mining can actually be reduced to a conformance checking problem.}\PopUpComment{Giacomo}{Some of the contents in here are good, and should be put elsewhere in the introduction. "In fact, despite these operators might be applied to query plans similarly to relational algebra operators, no work -- to the best of our knowledge -- exploited this possibility". But, this should be linked to another kind of problem, too!}
\medskip


%\paragraph*{Why do I want to talk about this problem? Why is it relevant?} \textit{Because current literature is lacking of a given aspect} 
%\section{Motivating Example}\label{sec:mot}
Correlations might be also exploited in real business use case scenarios: a goods brokerage company \cite{PetermannJMR14} trades items between producers (vendors) and retailers (customers): each activity starts with a vendor sending a sales quotation to a customer. Such a quotation considers both the item's purchase price as well as the additional revenues. The process stops when the customer rejects the offer. Otherwise, the order is confirmed, the item is scheduled for delivery and, when ready, is sent to a logistic operator. In this occasion, the sales invoice and the sales order is sent to the retailer. Next, both the producers and the retailers might rank the items on a scale from 0 to 10, where 0 denotes a despicable product while 10 denotes an excellent product. A retailer ranking a product extremely low can file a complaint ticket to the brokerage company which might grant a refund.
In this scenario, deviant traces are traces that either do not respect the company's rules, or traces that will potentially lead to retailers' complaints. %In particular, a company must send a product only after receiving the offer's acceptance. 
A business rule explicitly requiring $\Theta$-conditions is the following:   a late delivery complaints occur only if the date the product product is received is greater than the agreed time to deliver as registered in a previous agreement event. This situation cannot be directly expressed as a temporal pattern, as we also need to test the timestamps associated in the data payload. Albeit this task requires to represent temporal information within the data perspective \cite{MultiPerspective}, this would require to express \textit{correlation} conditions (\S\ref{sec:DAD}) within the single Declare template of interest. 


%\paragraph*{Who might be interested in our solution? How these people might use this work?} \textit{Please provide the pieces of information that are specific to your own research field, and provide some use case examples motivating the practicality of your approach}  

\RevRepl{Running temporal logical queries knowledge bases has many use-cases.}{Conformance checking can be applied to several other non-business domains:} 
 First, given that process model information can be exploited to represent the tasks performed by both physical and cybernetic agents \cite{Ioanna}, this information can be exploited to detect \textbf{cyber-security attacks}, where a model can be extracted from previous invasions allowing common patterns to be identified \cite{BENASHER201551,LagraaS20}. Then, the conformity of any trace to the model might be exploited for determining whether an attack occurred or not. Second, \RevRepl{Mining particular patterns within this data }{temporal models extracted from} hospital logs, consisting of diagnoses and treatments with their respective outcomes, could aid \textbf{healthcare} professionals in \RevDel{future} \textbf{decision making} \cite{Amantea2020}. \RevAdd{In fact, such models allow to associate a precondition to a consequence within a specific clinical event of interest, thus providing whitebox explainable AI} \cite{mining,KusumaKMHGJ20}. \RevRepl{One such study was performed by [Amantea2020], with results that could `demonstrate the ability to perform an healthcare process analysis'. In addition to the process discovery,}{As a result of the process discovery}, conformance checking tools might be exploited to assess whether the specific clinical case abides to the declarative rules associated to the model, thus allowing the prediction of a specific clinical event of interest. Last, most recent \textbf{videogames}  exploit AI features: existing state of the art already exploit automata \cite{Miyake2017} for modelling \textsc{Non Player Character}'s behaviours. As Declarative models and automata are completely equivalent approaches, developers might exploit the former to more compactly represent the latter. Furthermore, as debugging AI in videogames is a crucial challenge \cite{john2019debugging}, conformance checking solutions might be exploited for debugging unexpected behaviours in videogames. Furthermore, as AAA videogames already allow to track and log both players and NPC actions\footnote{\url{https://battlefieldtracker.com/}}, it might be also possible to use game logs for distinguishing winning strategies from losing ones \cite{mining}. As a result, analysis of an ongoing trace at runtime would then allow the model to `suggest' an action that is beneficial to the player based on the current state of the game, with the current strategy they are pursuing.

%\begin{itemize}
%	%\item A cyber-security attack, where a model can be extracted from previous invasions allowing common patterns to be identified. Some solutions such as \cite{BENASHER201551} use technologies to identify this, but they do not use conformance checking. 
%	\item A hospital log \cite{Amantea2020} [\dots]  \MarkText{In addition to the process discovery, these approaches are `opening the way to perform conformance checking and enhancement', which further justifies our argument that a process mining technique can be reduced to a conformance checking problem.} \PopUpComment{Giacomo}{The problem with this is that it does not explain how this could be done. We shall discuss this in person. Please see the above rephrasing.}
%	%\item Suggesting actions to players in video games. Conformance checking applications for AI in video games has not previously been research; existing state of the art \cite{Miyake2017} use either automata or machine learning. Process mining would allow models to be extracted that represent unique strategies players have attempted in the past. Information regarding their `success' can also be stored. As a result, analysis of an ongoing trace at runtime would then allow the model to `suggest' an action that is beneficial to the player based on the current state of the game, with the current strategy they are pursuing.
%\end{itemize}




%\paragraph*{What do I want to say (to the research community), precisely.} \textit{I want to communicate the general problem that I am aiming to solve} 
Current state of the art conformance checking solutions do not exploit the benefits of storing data in a custom relational database. When running queries, the same data is often accessed multiple times \cite{BurattinMS16,bpm21}. This is especially the case in the process of data-mining with large workloads \cite{SchonigRCJM16}, where the identification of patterns often share similar subqueries. On the other hand, Existing solutions \RevDel{, such as} \cite{BellatrecheKB21}\RevDel{, exploit this by} identify\RevDel{ing} \RevRepl{the}{common} sub\RevAdd{-}expressions within \RevRepl{a query}{several queries running contemporarely}\RevDel{ occurring more than once}, therefore \RevRepl{requiring only one computation}{reduce both the data access and the computation overhead to a minimum.} \RevAdd{This can be easily relate to the conformance checking problem, where multiple declarative clauses from the same model might be assessed contemporarily}. By decomposing these queries into \LTLf, a similar approach can also be followed. We propose that the queries, decomposed into \LTLf operators, can also follow a query plan similar to \cite{BellatrecheKB21}. We extend the approach by adapting the query plan to \RevRepl{use relational algebra with}{express \LTLf operators similarly to relational algebra operations}, where common sub-expressions can still be rationalised. Still, there is some prior work on \RevDel{Another approach proposes a solution that} decompos\RevRepl{es}{ing} clauses into traditional SQL queries \cite{SchonigRCJM16}. This solution \emph{does} exploits the benefits of using a relational database (and therefore query plans) by transforming declare clauses into traditional SQL queries. However, this solution is limited as it \RevRepl{does not}{neither} consider\RevAdd{s} data conditions (only event identifiers), \RevAdd{nor considers multiple clauses pertaining to disparate Declare templates}. This provides less functionality than we propose, where we are data-aware and theta conditions can be taken into account when performing any operators.



%Last, we can observe that some temporal information cannot be expressed by data-agnostic Declare templates. For example, a late delivery complaints occur if the date of a product is greater than the agreed time to deliver it in the previous sales order. This situation cannot be directly expressed as a $\textsf{Precedence}$, as we also need to test the timestamps as both data and event timestamps. Albeit this task requires to represent temporal information within the data perspective \cite{MultiPerspective}, this would require to express \textit{correlation} conditions (see \S\ref{ssec:dad}) within the single Declare template of interest. In the present work, we discard the possibility of expressing such correlation constraints: please observe that this is a quite common consideration within the spectrum of Business Process Management, and therefore we will continue to work under this working assumption \cite{10.1007/978-3-642-40176-3_8}. Nevertheless, we are planning to extend the proposed approach so to perform conformance checking containing correlation constraints. 
%Conformance checking is an integral part of artificial intelligence that bridges data mining and business process management. 


%Conformance checking can be extremely computationally intensive, both in time and storage, so optimised solutions are necessary to ensure a well performant implementation. To our knowledge, no solution existing whereby a relational base exploits optimised query plans, adapting solutions such as \cite{BellatrecheKB21}, in a business process environment using LTLf.
\medskip


\paragraph*{Now, communicate our idea also to the people working in our same area!} \textit{In particular, this means that we can go down in technicalities on what we want to solve, which are the primarily goals of our research, and which are the intermediate requirements/results leading to the results that we expect.} 

As process mining can be reduced to a conformance checking problem, a given log can be queried against a declarative model at runtime, and the same conformance checking calculations can be applied to generate its conformance \emph{at the current time}. Therefore, KnoBAB provides an optimized representation of the trace logs over which the declarative models $\mathcal{M}$ are going to be both queried and mined with \LTLf.

We propose a knowledge base, KnoBAB, which provides efficient conformance checking by adapting query plan optimisations \cite{BellatrecheKB21} to \LTLf. In addition, we provide data-aware capabilities, which discussed database solutions do not. KnoBAB provides the conformance of a \emph{trace} to a set of clauses, not the conformance of a clause against a log. This is more valuable in scenarios where trace information could point to where, and why, it was a deviant. Such knowledge could then be used for many features, such as generating the repair for this trace.
\medskip
 

The greatest amount of performance gain is due to the custom query plan, structured in such a way that multiple queries are stored within a graph, and then batch jobs are run using \textbf{parallelisation}. When process mining, large numbers of queries are performed, therefore there will be many instances of duplicate data accessing, resulting in poor optimisation. In this approach, there is the guarantee that unique data elements are obtained and processed only once, while current state of the art process-mining approaches access data per query. 
