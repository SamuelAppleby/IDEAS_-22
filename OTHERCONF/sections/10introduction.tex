\section{Introduction}

\paragraph*{Brief introduction of your general area of interest: provide the \textbf{context} to the overall general setting.}
\textit{Conformance checking} is a well-known \textsc{Process Mining} technique determining if a sequence of distinguishable events (i.e., a \textit{trace}) conforms to the expected process behavior represented as a \textit{process model} \cite{RozinatA08}. Such a model might be either represented as a set of temporal clauses, determining correlations between events happening at a previous time of the trace (\textit{activation}) and others happening in the immediate future (\textit{target}). Such temporal rules are not limited to the mere presence of specific events within the trace, but also determine specific patterns how such clauses might occur (\S\ref{sec:DAD}). When a trace does not adhere to the model, we say that the trace is \textit{deviant} \cite{bpm21}.
\medskip

\paragraph*{What do I want to say (to the research community), precisely.} \textit{I want to communicate the general problem that I am aiming to solve}
\medskip

\paragraph*{Why do I want to talk about this problem? Why is it relevant?} \textit{Because current literature is laking of a given aspect}
\medskip

\paragraph*{Who might be interested in our solution? How these people might use this work?} \textit{Please provide the pieces of information that are specific to your own research field, and provide some use case examples motivating the practicality of your approach} Therefore, the system is adaptable for real-time applications: a video game run, where repairs can be analysed to provide \emph{suggestions} to the player; events leading up to a cyber-security attack, where repairs can provide advice on how to prevent the attack progressing etc. \texttt{\color{red}[TODO]}
\medskip


\paragraph*{Now, communicate our idea also to the people working in our same area!} \textit{In particular, this means that we can go down in technicalities on what we want to solve, which are the primarily goals of our research, and which are the intermediate requirements/results leading to the results that we expect.} 
In this paper, we propose the addition of a knowledge base (KB) to providing an optimized representation of the trace logs over which the declarative models $\mathcal{M}$ are going to be queried.\texttt{\color{red}[TODO]} % By storing it in a KB,  data is obtained and processed only once, rather than per query, as existing state of the art. 
